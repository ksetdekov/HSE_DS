\documentclass[a4paper,12pt]{article}
\usepackage[utf8]{inputenc}
\usepackage[cm,empty]{fullpage}
\usepackage[T2A]{fontenc}
\usepackage[english, russian]{babel}
\usepackage{amssymb,amsmath,amsxtra,amsthm}
\usepackage{proof}
\usepackage[pdftex]{graphicx}
\usepackage{wrapfig}

\usepackage[left=2cm,right=2cm,
    top=1cm,bottom=1cm,bindingoffset=0cm]{geometry}

\renewcommand{\leq}{\leqslant}
\renewcommand{\geq}{\geqslant}


\newcommand{\iiff}{\Longleftrightarrow}
\renewcommand{\iff}{\Leftrightarrow}
\newcommand{\nothing}{\varnothing}


\newcommand{\NN}{\mathbb{N}}
\newcommand{\ZZ}{\mathbb{Z}}
\newcommand{\Q}{\mathbb{Q}}
\newcommand{\A}{\mathbb{A}}
\newcommand{\R}{\mathbb{R}}
\renewcommand{\C}{\mathbb{C}}

\renewcommand{\phi}{\varphi}
\newcommand{\eps}{\varepsilon}

\newcounter{z}


\newcommand{\zs}{\refstepcounter{z}\vskip 10pt\par\noindent
\fbox{\textbf{12.\arabic{z}}} }

\newcommand{\z}{\refstepcounter{z}\vskip 20pt\noindent
\fbox{\textbf{\arabic{z}}} }

\renewcommand{\date}{{\bf 20 января 2021}} %Дата занятия

\newcommand{\dif}
{
------------------------------------------------------------------------------------------------------------------------------------------------------
}

\newcommand{\HSEhat}{
\vspace*{-0pt}
\noindent
\setcounter{z}{0}
\vspace*{-10pt}
\begin{wrapfigure}[2]{l}{5pt}
\vspace*{-25pt}
\hspace*{-20pt}
\includegraphics[scale=0.18]{img/HSE_LOGO.png}
\end{wrapfigure}
\vspace{-20pt}


{\bf \phantom{\date}  \large \hfill Дискретная математика: \hfill \normalsize \date}

\vspace{5 pt}
{\bf \large \hfill  комбинаторика и вероятность.\hfill }

\vspace{15 pt}
\centerline{ \large  Домашнее задание.}



\vspace*{10pt}
\setcounter{z}{0}

}

\begin{document}

\HSEhat

\z После опроса $250$ человек оказалось, что английский знают ровно $210$ респондентов, испанский --- $100$, а оба языка --- $80$. Сколько из опрошенных не знают ни английского, ни испанского? 

\z Есть $10$ кандидатов на $6$ различных вакансий. Каждого кандидата можно взять на любую вакансию. Сколькими способами можно заполнить вакансии? (Каждая вакансия должна быть заполнена ровно одним человеком.

\z Найдите вероятность того, что в случайном шестизначном коде будет хотя бы две одинаковые цифры.

\z {\bf a)} Каких натуральных чисел больше среди первого миллиона: тех, в записи которых
есть единица или тех, в записи которых её нет?

{\bf б)} Тот же вопрос для первых 10 миллионов чисел.\\

(В нашем курсе мы считаем, что натуральные числа начинаются с $0$.)

\z Найдите вероятность выпадения дубля при броске двух кубиков (дубль означает, что на обоих кубиках выпало одинаковое значение).

%\z Найдите вероятность того, что случайно выбранное число от 1 до 100 делится на 9.

\z Команда принимает участие в турнире, где сыграет {\it четыре} игры.

Вероятность выиграть в первом матче равна $\frac 12$. Вероятность выигрыша после победы в предыдущем матче возрастает до $\frac 23$, а после поражения уменьшается до $\frac 13$.

Какова вероятность

{\bf a)} выиграть не менее двух игр?

{\bf б)} выиграть ровно две игры?

%{\bf в)} выиграть две игры подряд?

\z Монету бросают восемь раз. Найдите вероятности событий:

{\bf a)} $A$ --- "орел выпал 6 раз";

%{\bf б)} $B$ --- "орел выпал не более двух раз";

{\bf б)} $B$ --- "орел выпал не менее трех раз".

%\z Василий Петров выполняет задание по английскому языку. В этом задании есть 10 английских выражений и их переводы на русский в случайном порядке. Нужно установить верные соответствия между выражениями и их переводами. За каждое правильно установленное соответствие даётся 1 балл. Таким образом, можно получить от 0 до 10 баллов. Вася ничего не знает, поэтому выбирает варианты наугад. Найдите вероятность того, что он получит ровно 9 баллов.


\end{document}


\z Вероятность искажения одного знака при передаче сообщения равна $\frac{1}{10}$. Найдите вероятности событий:

{\bf a)} $A$ --- "сообщение из 5 знаков не будет искажено";

{\bf б)} $B$ --- "сообщение из 5 знаков содержит ровно одно искажение";

{\bf в)} $C$ --- "сообщение из 5 знаков содержит не более трех искажений".
