\documentclass[a4paper,12pt]{article}
\usepackage[utf8]{inputenc}
\usepackage[cm,empty]{fullpage}
\usepackage[T2A]{fontenc}
\usepackage[english, russian]{babel}
\usepackage{amssymb,amsmath,amsxtra,amsthm}
\usepackage{proof}
\usepackage[pdftex]{graphicx}
\usepackage{wrapfig}
\usepackage{braket}
\usepackage{xcolor}
\usepackage{enumitem}

\usepackage[left=2cm,right=2cm,
    top=1cm,bottom=1cm,bindingoffset=0cm]{geometry}

\renewcommand{\leq}{\leqslant}
\renewcommand{\geq}{\geqslant}


\newcommand{\iiff}{\Longleftrightarrow}
\renewcommand{\iff}{\Leftrightarrow}
\newcommand{\nothing}{\varnothing}

\newtheorem*{rem}{Замечание}

\newcommand{\NN}{\mathbb{N}}
\newcommand{\ZZ}{\mathbb{Z}}
\newcommand{\Q}{\mathbb{Q}}
\newcommand{\A}{\mathbb{A}}
\newcommand{\R}{\mathbb{R}}
\renewcommand{\C}{\mathbb{C}}

\renewcommand{\phi}{\varphi}
\newcommand{\eps}{\varepsilon}

\makeatletter
\newcommand*{\rom}[1]{\expandafter\@slowromancap\romannumeral #1@}
\makeatother

\newcounter{z}


\newcommand{\zs}{\refstepcounter{z}\vskip 10pt\par\noindent
\fbox{\textbf{12.\arabic{z}}} }

\newcommand{\z}{\refstepcounter{z}\vskip 20pt\noindent
\fbox{\textbf{\arabic{z}}} }

\renewcommand{\date}{{\bf 27 мая 2021}} 

\newcommand{\dif}
{
------------------------------------------------------------------------------------------------------------------------------------------------------
}

\newcommand{\HSEhat}{
\vspace*{-0pt}
\noindent
\setcounter{z}{0}


{\bf \phantom{\date}  \large \hfill Теория вероятностей: \hfill \normalsize \date}

\vspace{5 pt}
{\bf \large \hfill  лекция 4\hfill }

\vspace{15 pt}
\centerline{ \large  Домашнее задание.}
\centerline{ \large  Кирилл Сетдеков}



\vspace*{10pt}
\setcounter{z}{0}

}

\begin{document}
\HSEhat


\begin{enumerate}

\subsection*{Задачи:}


\item	При $\lambda > 0$ определим плотность f так: \\
\includegraphics[width=6cm]{img/task1.png}

Такая функция f называется двусторонней экспоненциальной: пусть величина X имеет плотность f. Найдите плотность случайной величины |X|. [Указание. Вычислите сначала функцию распределения.]
\item	С вероятностью попадания при одном выстреле 0,7 охотник стреляет по дичи до первого попадания, но успевает сделать не более 4 выстрелов. Пусть X - число промахов. Построить таблицу и функцию распределения. 
\item	Случайная величина X равномерно распределена на отрезке [0, 1]. Докажите, что случайная величина:
 
$$Y = -\frac{1}{\lambda}\ln{(1-X)}$$
имеет экспоненциальное распределение с параметром $\lambda > 0$. Этот факт может использоваться для генерации экспоненциального распределения с помощью равномерного: чтобы сгенерировать экспоненциальную случайную величину можно сгенерировать равномерно распределенную на [0, 1] случайную величину и применить к ней выше приведенное преобразование. 
\item	 Случайные приращения цен акций двух компаний за день X и Y имеют совместное распределение, заданное таблицей:
\begin{center}
 \begin{tabular}{|c| c| c |} 
 \hline
 $X$ \slash $Y$ & $-1$& $+1$ \\ [0.5ex] 
 \hline
 $-1$ & 0,3 & 0,2 \\ 
 \hline
 $+1$ & 0,1	 & 0,4  \\ 
 \hline
\end{tabular}
\end{center}

Найти коэффициент корреляции.

\textbf{Решение:}\\

Для удобства расчета, дополнительно запишем в таблицу вероятности $p(XY), p(X), p(Y)$ и значения $X^2, Y^2$:

\begin{center}
 \begin{tabular}{|c|c|c| c| c |} 
 \hline
 Значение & $p(XY)$& $p(X)$& $p(Y)$&$X^2; Y^2$ \\ [0.5ex] 
 \hline
 $-1$ & 0,3 & 0,5&0,4&1 \\ 
 \hline
 $+1$ & 0,7	 & 0,5&0,6&1  \\ 
 \hline
\end{tabular}
\end{center}
$E(X) = -1 \cdot 0.5 + 1\cdot 0.5=0$\\
$E(Y) = -1 \cdot 0.4 + 1\cdot 0.6=0.2$\\
$E(XY) = -1 \cdot 0.3 + 1\cdot 0.7=0.2$. Вероятности для $XY$ найдены из суммирования вероятностей тех исходов, которые дают произведения $-1$ и $+1$\\
$E(X^2) = 1 \cdot 0.5 + 1\cdot 0.5=1$\\
$E(Y^2) = 1 \cdot 0.4 + 1\cdot 0.6=1$\\

$D(X) = E(X^2)- [E(X)]^2=1-0 = 1$\\
$D(Y) = E(Y^2)- [E(Y)]^2=1-0.04 = 0.96$\\
$Cov(X,Y)=E(XY)-E(X)E(Y)=0.4-0\cdot0.2=0.4$

$$Corr(X,Y) = \frac{Cov(X,Y)}{\sqrt{D(X)D(Y)}}=\frac{0.4}{\sqrt{1\cdot0.96}}\approx 0.40825$$

\textbf{Ответ: Корреляция акций: $Corr(X,Y)\approx 0.40825$} 



\item	Симметричная игральная кость имеет две зеленые, две красные и две белые стороны. Ее подбрасывают один раз. Пусть X = 1, если зеленая сторона сверху и 0 иначе. А Y=1, если красная сторона сверху и 0 иначе. Найдите ковариацию и корреляцию этих случайных величин

\textbf{Решение:}\\
Для удобства расчета, дополнительно запишем в таблицу вероятности $p(X), p(Y)$ и значения $X^2, Y^2, X, Y$:

\begin{center}
 \begin{tabular}{|c|c|c| c| } 
 \hline
 Грань & Вероятность& $X; X^2$& $Y; Y^2$ \\ [0.5ex] 
 \hline
 зеленый & 1/3 & 1&0 \\ 
 \hline
 красный & 1/3	 & 0&1  \\ 
 \hline
 белый& 1/3	 & 0&0  \\ 
 \hline
\end{tabular}
\end{center}

Вероятности выпадения X и Y равны, как и их значения, следовательно:
$E(X)=E(Y)=1\cdot1/3+0\cdot2/3=1/3$\\
$XY=0 \Rightarrow E(XY)=0$\\
$E(X^2)=E(Y^2)=1\cdot1/3+0\cdot2/3=1/3$\\
$D(X)=D(Y)=E(X^2)-[E(X)]^2=\frac{1}{3}-\frac{1}{9}=\frac{2}{9}$\\
$Cov(X,Y)=E(XY)-E(X)E(Y)=0-\frac{1}{3}\frac{1}{3}=-\frac{1}{9}$

$$Corr(X,Y) = \frac{Cov(X,Y)}{\sqrt{D(X)D(Y)}}=\frac{-\frac{1}{9}}{\sqrt{\frac{2}{9}\frac{2}{9}}}=-\frac{\frac{1}{9}}{\frac{2}{9}}=-0.5$$

\textbf{Ответ: $Cov(X,Y)=-\frac{1}{9}\approx-0.11$; $Corr(X,Y)=-0.5$} 

\item	Найдите коэф. Ковариации и корреляции для двух независимых случайных величин X ~ Exp(3) и X ~ Exp(12)

\textbf{Решение:}\\
\textbf{Ответ: } 

\end{enumerate}
\end{document}