\documentclass[a4paper,12pt]{article}
\usepackage[utf8]{inputenc}
\usepackage[cm,empty]{fullpage}
\usepackage[T2A]{fontenc}
\usepackage[english, russian]{babel}
\usepackage{amssymb,amsmath,amsxtra,amsthm}
\usepackage{proof}
\usepackage[pdftex]{graphicx}
\usepackage{wrapfig}
\usepackage{braket}
\usepackage{xcolor}
\usepackage{enumitem}

\usepackage[left=2cm,right=2cm,
    top=1cm,bottom=1cm,bindingoffset=0cm]{geometry}

\renewcommand{\leq}{\leqslant}
\renewcommand{\geq}{\geqslant}


\newcommand{\iiff}{\Longleftrightarrow}
\renewcommand{\iff}{\Leftrightarrow}
\newcommand{\nothing}{\varnothing}

\newtheorem*{rem}{Замечание}

\newcommand{\NN}{\mathbb{N}}
\newcommand{\ZZ}{\mathbb{Z}}
\newcommand{\Q}{\mathbb{Q}}
\newcommand{\A}{\mathbb{A}}
\newcommand{\R}{\mathbb{R}}
\renewcommand{\C}{\mathbb{C}}

\renewcommand{\phi}{\varphi}
\newcommand{\eps}{\varepsilon}

\makeatletter
\newcommand*{\rom}[1]{\expandafter\@slowromancap\romannumeral #1@}
\makeatother

\newcounter{z}


\newcommand{\zs}{\refstepcounter{z}\vskip 10pt\par\noindent
\fbox{\textbf{12.\arabic{z}}} }

\newcommand{\z}{\refstepcounter{z}\vskip 20pt\noindent
\fbox{\textbf{\arabic{z}}} }

\renewcommand{\date}{{\bf 15 мая 2021}} 

\newcommand{\dif}
{
------------------------------------------------------------------------------------------------------------------------------------------------------
}

\newcommand{\HSEhat}{
\vspace*{-0pt}
\noindent
\setcounter{z}{0}


{\bf \phantom{\date}  \large \hfill Теория вероятностей: \hfill \normalsize \date}

\vspace{5 pt}
{\bf \large \hfill  лекция 1\hfill }

\vspace{15 pt}
\centerline{ \large  Домашнее задание.}
\centerline{ \large  Кирилл Сетдеков}



\vspace*{10pt}
\setcounter{z}{0}

}

\begin{document}
\HSEhat


\begin{enumerate}

\subsection*{Задачи:}



\item Из колоды вынимается карта. Если это пики, то эксперимент прекращается. Если нет, то карта откалывается и вынимается новая. Опишите вероятностное пространство. Необязательно перечислять все элементы пространства, достаточно описать общую идею\\
\textbf{Решение:}\\
Вероятностное пространство будет состоять из элементарных исходов, число которых равно числу карт в колоде (положим стандартная колода, 52 карты, $|\Omega|=52$).\\
Элементарные исходы будут:\\
$a_1:$ мы достали пиковую карту на 1 шаге. $P(a_1)=\frac{1}{4}$\\
$a_2:$ мы достали пиковую карту на 2 шаге, но не достали на прошлом шаге. $P(a_2)=(1-P(a_1))\frac{13}{52-1}=\frac{13\cdot3}{4\cdot51} \approx 0.19$\\
$a_k:$ мы достали пиковую карту на k шаге, но не достали на всех прошлых (k-1) шагах. $P(a_k)=(1-\sum_{k=1}^{k-1}P(a_i))\frac{13}{52-k+1}=\frac{13\cdot3}{4\cdot51} \approx 0.19$\\
$a_{40}:$ мы достали пиковую карту на 40 шаге, но не достали на всех прошлых (39) шагах. $P(a_k)=(1-\sum_{k=1}^{39}P(a_i))\frac{13}{13}=\frac{13\cdot3}{4\cdot51} \approx 1.57 \times 10^{-12}$\\
$a_{m}, m>40:$ мы достали пиковую карту на m шаге. $P(a_m)=0$ так как мы точно достали пиковую карту на шаге 40 или раньше. 




\item Аналогичный эксперимент, только карта возвращается обратно в колоду, и колода тасуется каждый раз. Опишите вероятностное пространство

\textbf{Решение:}\\
Вероятностное пространство будет состоять из элементарных исходов, число которых бесконечно (так как мы можем сколько угодно удачно доставать не пиковую карту).\\
Элементарные исходы будут:\\
$a_1:$ мы достали пиковую карту на 1 шаге. $P(a_1)=\frac{1}{4}$\\
$a_2:$ мы достали пиковую карту на 2 шаге, но не достали на прошлом шаге. $P(a_2)=(1-P(a_1))\frac{1}{4}=\frac{3}{16}$\\
$a_k:$ мы достали пиковую карту на k шаге, но не достали на всех прошлых (k-1) шагах. $P(a_k)=(1-P(a_1))^{(k-1)}\frac{1}{4}=(\frac{3}{4})^{(k-1)}\frac{1}{4}$

\item Мастер, имея 20 деталей, из которых 4 – нестандартных, проверяет детали одну за другой, пока ему не попадется стандартная. Какова вероятность, что он проверит ровно две детали? 

\textbf{Решение:}\\
Для того чтобы проверить только 2 детали (событие $A$), он должен проверить сначала нестандартную деталь (4 из 20), а потом взять стандартную (16 из 19 оставшихся). $P(A) =  \frac{4}{20} \frac{16}{19}=\frac{16}{95}\approx0.1684$ 

\textbf{Ответ: вероятность, что он проверит ровно две детали: $\frac{16}{95}\approx0.1684$}

\item Игральная кость бросается трижды. Какова вероятность того, что с каждым разом вы получаете все больший номер? 

\textbf{Решение:}\\
Выпишем подходящие комбинации.
Нам подойдет любой из вариантов последовательностей результатов из этого набора: \\
$A = \{(1,2,3), (1,2,4), (1,2,5), (1,2,6), (1,3,4), (1,3,5), (1,3,6), (1,4,5), (1,4,6), (1,5,6), \\(2,3,4), (2,3,5), (2,3,6), (2,4,5), (2,4,6), (2,5,6), (3,4,5), (3,4,6), (3,5,6), (4,5,6)\}$

$$|A| = 20$$
Пространство элементарных исходов состоит из 3 выпавших значений на 3 кубиках и имеет элементов:
$$|\Omega|=6^3=216$$

$P(A) = \frac{|A|}{|\Omega|}=\frac{20}{216}=\frac{5}{54}\approx0.0926$

\textbf{Ответ: вероятность того, что с каждым разом вы получаете все больший номер из трех бросков: $\frac{5}{54}\approx0.0926$}


\item В электронном приборе имеются лампы двух типов. Прибор не работает тогда и только тогда, когда есть бракованные лампы обоих типов. Вероятность того, что бракованы лампы первого типа, равна 0.1, второго типа — 0.2. Известно, что две лампы бракованы. Какова вероятность того, что, несмотря на это, прибор работает?


\textbf{Решение:}\\
Интерпретируем задание: у нас есть лампы типа $A$, для каждой из них есть вероятность $0.1$, что она сломана и типа $B$, для каждой из них есть вероятность $0.2$, что она сломана. Вероятность получить лампу A или B одинаковая. $w$ - прибора работает.
Из двух сломанных ламп может быть 4 варианта, какие нам попались сломанные.\\
$P(AA) = 0.1\cdot0.1 = 0.01$ Прибор при этом работает: $P(w|AA) = 1$\\
$P(AB) = 0.1\cdot0.2 = 0.02$ Прибор при этом сломан: $P(w|AB) = 0$\\
$P(BA) = 0.2\cdot0.1 = 0.02$ Прибор при этом сломан: $P(w|BA) = 0$\\
$P(BB) = 0.2\cdot0.2 = 0.04$ Прибор при этом работает: $P(w|BB) = 1$\\
Найдем вероятность того, что устройство работает через полную вероятность:

\begin{align*}P(w)=P(w|AA)P(AA)+P(w|AB)P(AB)+P(w|BA)P(BA)+P(w|BB)P(BB)=\\=0.01\cdot1+0.02\cdot0+0.02\cdot0+0.04\cdot1=0.05\end{align*}

$$P(2broken)= P(AA)+P(AB)+P(BA)+P(BB)=0.09$$

$$P(w|2broken)= \frac{P(w)}{P(2broken)}=\frac{0.05}{0.09}=\frac{5}{9} \approx 0.556$$

\textbf{Ответ: вероятность того, что, c 2 бракованными лампами прибор работает: $\frac{5}{9} \approx 0.556$ }





\item Сначала бросается идеальная игральная кость, затем — симметричные монеты в количестве, равном номеру, выпавшему на кости. Какова вероятность получить в результате 6 «орлов»? А 1 «орел»? 

\textbf{Решение:}\\
Вариант получить 6 орлов только 1: выбросить "6", потом выбросить 6 орлов. Вероятность этого события $P = \frac{1}{6}(\frac{1}{2})^6=\frac{1}{384}\approx0.0026$\\
Вариант получить 1 орел без решек только 1: выбросить "1", потом выбросить 1 орел. Вероятность этого события $P = \frac{1}{6}\frac{1}{2}=\frac{1}{12}\approx0.083$\\

Вариантов выбросить 1 орла с любым количеством решек несколько, в зависимости от броска кубика:\\
Выпало 1: $P(O|1) = \frac{1}{2}$\\
Выпало 2: 1 орел - случайная величина, распределенная по Бернулли c 2 испытаниями и 1 успехом: $P(O|2) = C^1_2 \frac{1}{2} \frac{1}{2} = \frac{1}{2}$\\
Выпало 3: 1 орел - случайная величина, распределенная по Бернулли c 3 испытаниями и 1 успехом: $P(O|3) = C^1_3 \frac{1}{2} {(\frac{1}{2})}^2 = \frac{3}{8}$\\
Выпало 4: 1 орел - случайная величина, распределенная по Бернулли c 4 испытаниями и 1 успехом: $P(O|4) = C^1_4 \frac{1}{2} {(\frac{1}{2})}^3 = 4\frac{1}{16}=\frac{1}{4}$\\
Выпало 5: 1 орел - случайная величина, распределенная по Бернулли c 5 испытаниями и 1 успехом: $P(O|5) = C^1_5 \frac{1}{2} {(\frac{1}{2})}^4 = 5\frac{1}{32}=\frac{5}{32}$\\
Выпало 6: 1 орел - случайная величина, распределенная по Бернулли c 6 испытаниями и 1 успехом: $P(O|6) = C^1_6 \frac{1}{2} {(\frac{1}{2})}^5 = 6\frac{1}{64}=\frac{3}{32}$\\

Найдем вероятность 1 орла через полную вероятность:
\begin{align*}P(O) = P(O|1)P(1) +P(O|2)P(2)+P(O|3)P(3)+P(O|4)P(4)+P(O|5)P(5)+P(O|6)P(6)=\\=\frac{1}{2}\frac{1}{6}+\frac{1}{2}\frac{1}{6}+\frac{3}{8}\frac{1}{6}+\frac{1}{4}\frac{1}{6}+\frac{5}{32}\frac{1}{6}+\frac{3}{32}\frac{1}{6}=\frac{5}{16}=0.3125\end{align*}


\textbf{Ответ: вероятность получить в результате 6 «орлов»: $\frac{1}{384}\approx0.0026$. 1 «орел» без других монет: $\frac{1}{12}\approx0.083$. 1 «орел» с любым числом решек или без решек: $\frac{5}{16}=0.3125$ }

\item Вам сообщили, что из четырех карт, лежащих на столе рубашкой вверх, две принадлежат красной масти и две — черной. Если вы называете цвет наудачу, то с какой вероятностью дадите 0, 2, 4 верных ответа?

\textbf{Решение:}\\
У нас для каждой карты шанс ответить верно $p=0.5$. Вероятность ответить корректно k раз это с.в X, имеющая распределение Бернулли.

$$P(x=0) = C_4^0 p^0 q^4 = \frac{1}{2^4}=0.0625$$
$$P(x=2) = C_4^2 p^2 q^2 = 6 \frac{1}{2^4}=0.375$$
$$P(x=4) = C_4^4 p^4 q^0 = \frac{1}{2^4}=0.0625$$

\textbf{Ответ: вероятность дать 0 верных ответов: $0.0625$. Вероятность дать 2 верных ответа: $0.375$. Вероятность дать 4 верных ответа: $0.0625$}

\item В телеграфном сигнале «точки» и «тире» появляются в пропорции 3:4. В силу некоторых причин, вызывающих очень сильное искажение сообщения, «точки» заменяются на «тире» с вероятностью 1/4, в то время как «тире» превращаются в «точки» с вероятностью 1/3. Если получена «точка», то какова вероятность того, что этот символ был «точкой» при отправлении сообщения?

\textbf{Решение:}\\
Обозначим $H_1$ - исходная точка, $H_2$ - исходное тире.
По условию: $P(H_1)=3/7$, $P(H_2)=4/7$\\
Используя условие, обозначим условные вероятности: $$P(\cdot|H_1)=3/4$$
$$P(-|H_2)=2/3$$
Так как нас интересуют точки, обозначим обратное событие к получению тире при отправке тире $ \backslash  {-|H_2}=\cdot|H_2$
$$P(\cdot|H_2)=1-P(-|H_2)=1/3$$

Запишем полную вероятность для получения точки:
$$P(\cdot) = P(\cdot|H_1)P(H_1)+P(\cdot|H_2)P(H_2) = \frac{3}{4}\frac{3}{7}+\frac{1}{3}\frac{4}{7}=\frac{43}{84}$$

По формуле Байеса, найдем ответ на вопрос в задании:
$$P(H_1|\cdot) = \frac{P(\cdot|H_1)P(H_1)}{P(\cdot)}=\frac{\frac{3}{4}\frac{3}{7}}{\frac{43}{84}}=\frac{27}{43}\approx0.628$$

\textbf{Ответ: Если получена «точка», то вероятность того, что этот символ был «точкой» при отправлении сообщения $\frac{27}{43}\approx0.628$}

\end{enumerate}
\end{document}