\documentclass[a4paper,12pt]{article}
\usepackage[utf8]{inputenc}
\usepackage[cm,empty]{fullpage}
\usepackage[T2A]{fontenc}
\usepackage[english, russian]{babel}
\usepackage{amssymb,amsmath,amsxtra,amsthm}
\usepackage{proof}
\usepackage[pdftex]{graphicx}
\usepackage{wrapfig}
\usepackage{braket}
\usepackage{xcolor}
\usepackage{enumitem}

\usepackage[left=2cm,right=2cm,
    top=1cm,bottom=1cm,bindingoffset=0cm]{geometry}

\renewcommand{\leq}{\leqslant}
\renewcommand{\geq}{\geqslant}


\newcommand{\iiff}{\Longleftrightarrow}
\renewcommand{\iff}{\Leftrightarrow}
\newcommand{\nothing}{\varnothing}

\newtheorem*{rem}{Замечание}

\newcommand{\NN}{\mathbb{N}}
\newcommand{\ZZ}{\mathbb{Z}}
\newcommand{\Q}{\mathbb{Q}}
\newcommand{\A}{\mathbb{A}}
\newcommand{\R}{\mathbb{R}}
\renewcommand{\C}{\mathbb{C}}

\renewcommand{\phi}{\varphi}
\newcommand{\eps}{\varepsilon}

\makeatletter
\newcommand*{\rom}[1]{\expandafter\@slowromancap\romannumeral #1@}
\makeatother

\newcounter{z}


\newcommand{\zs}{\refstepcounter{z}\vskip 10pt\par\noindent
\fbox{\textbf{12.\arabic{z}}} }

\newcommand{\z}{\refstepcounter{z}\vskip 20pt\noindent
\fbox{\textbf{\arabic{z}}} }

\renewcommand{\date}{{\bf 14 мая 2021}} 

\newcommand{\dif}
{
------------------------------------------------------------------------------------------------------------------------------------------------------
}

\newcommand{\HSEhat}{
\vspace*{-0pt}
\noindent
\setcounter{z}{0}


{\bf \phantom{\date}  \large \hfill Теория вероятностей: \hfill \normalsize \date}

\vspace{5 pt}
{\bf \large \hfill  лекция 1\hfill }

\vspace{15 pt}
\centerline{ \large  Домашнее задание.}
\centerline{ \large  Кирилл Сетдеков}



\vspace*{10pt}
\setcounter{z}{0}

}

\begin{document}
\HSEhat


\begin{enumerate}

\subsection*{Задачи:}



\item Из колоды вынимается карта. Если это пики, то эксперимент прекращается. Если нет, то карта откалывается и вынимается новая. Опишите вероятностное пространство. Необязательно перечислять все элементы пространства, достаточно описать общую идею

\item Аналогичный эксперимент, только карта возвращается обратно в колоду, и колода тасуется каждый раз. Опишите вероятностное пространство
\item Мастер, имея 20 деталей, из которых 4 – нестандартных, проверяет детали одну за другой, пока ему не попадется стандартная. Какова вероятность, что он проверит ровно две детали? 
\item Игральная кость бросается трижды. Какова вероятность того, что с каждым разом вы получаете все больший номер? 
\item В электронном приборе имеются лампы двух типов. Прибор не работает тогда и только тогда, когда есть бракованные лампы обоих типов. Вероятность того, что бракованы лампы первого типа, равна 0.1, второго типа — 0.2. Известно, что две лампы бракованы. Какова вероятность того, что, несмотря на это, прибор работает?
\item Сначала бросается идеальная игральная кость, затем — симметричные монеты в количестве, равном номеру, выпавшему на кости. Какова вероятность получить в результате 6 «орлов»? А 1 «орел»? 
\item Вам сообщили, что из четырех карт, лежащих на столе рубашкой вверх, две принадлежат красной масти и две — черной. Если вы называете цвет наудачу, то с какой вероятностью дадите 0, 2, 4 верных ответа?
\item В телеграфном сигнале «точки» и «тире» появляются в пропорции 3:4. В силу некоторых причин, вызывающих очень сильное искажение сообщения, «точки» заменяются на «тире» с вероятностью 1/4, в то время как «тире» превращаются в «точки» с вероятностью 1/3. Если получена «точка», то какова вероятность того, что этот символ был «точкой» при отправлении сообщения?


\item Найти дисперсию и МО суммы очков, выпавших на n игральных костях\\
\textbf{Решение:}\\
Пусть $a$ - случайная величина, которая задает значение, которое выдает 1 кубик. Запишем ее значения, вероятности и $a^2$, чтобы посчитать $E(a); D(a)$
\begin{center}
 \begin{tabular}{|c| c| c| c|c|c|c|} 
 \hline
 $a$ & 1 & 2 & 3 & 4 & 5& 6\\ 
 \hline
  $a^2$ & 1 & 4 & 9 & 16 & 25& 36\\ 
 \hline
$P$ & 1/6 & 1/6 & 1/6 & 1/6 & 1/6& 1/6\\ 
\hline
\end{tabular}
\end{center}
$$E(a)=\sum_i^{6} {a_i P_i} = \frac{28}{6}=\frac{7}{2}=3.5$$
$$D(a)=E(a^2) - (E(a))^2=\sum_i^{6} {a^2_i P_i}-\frac{7^2}{2^2} = \frac{91}{6}-\frac{49}{4}=\frac{35}{12}=2\frac{11}{12}$$

По свойству МО: $E(n a) = nE(a) = \frac{7}{2}n$\\
По свойству дисперсии, учитывая, что результаты кубиков независимы:\\ $D(n a) = n^2E(a) = 2\frac{11}{12}n^2$\\
\textbf{Ответ: Дисперсия: $2\frac{11}{12}n^2$, МО: $3\frac{1}{2}n$}


\end{enumerate}
\end{document}