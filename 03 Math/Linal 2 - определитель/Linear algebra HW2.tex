\documentclass[a4paper,12pt]{article}
\usepackage[utf8]{inputenc}
\usepackage[cm,empty]{fullpage}
\usepackage[T2A]{fontenc}
\usepackage[english, russian]{babel}
\usepackage{amssymb,amsmath,amsxtra,amsthm}
\usepackage{proof}
\usepackage[pdftex]{graphicx}
\usepackage{wrapfig}
\usepackage{braket}
\usepackage{xcolor}

\usepackage[left=2cm,right=2cm,
    top=1cm,bottom=1cm,bindingoffset=0cm]{geometry}

\renewcommand{\leq}{\leqslant}
\renewcommand{\geq}{\geqslant}


\newcommand{\iiff}{\Longleftrightarrow}
\renewcommand{\iff}{\Leftrightarrow}
\newcommand{\nothing}{\varnothing}

\newtheorem*{rem}{Замечание}

\newcommand{\NN}{\mathbb{N}}
\newcommand{\ZZ}{\mathbb{Z}}
\newcommand{\Q}{\mathbb{Q}}
\newcommand{\A}{\mathbb{A}}
\newcommand{\R}{\mathbb{R}}
\renewcommand{\C}{\mathbb{C}}

\renewcommand{\phi}{\varphi}
\newcommand{\eps}{\varepsilon}

\makeatletter
\newcommand*{\rom}[1]{\expandafter\@slowromancap\romannumeral #1@}
\makeatother

\newcounter{z}


\newcommand{\zs}{\refstepcounter{z}\vskip 10pt\par\noindent
\fbox{\textbf{12.\arabic{z}}} }

\newcommand{\z}{\refstepcounter{z}\vskip 20pt\noindent
\fbox{\textbf{\arabic{z}}} }

\renewcommand{\date}{{\bf 22 марта 2021}} 

\newcommand{\dif}
{
------------------------------------------------------------------------------------------------------------------------------------------------------
}

\newcommand{\HSEhat}{
\vspace*{-0pt}
\noindent
\setcounter{z}{0}


{\bf \phantom{\date}  \large \hfill Линейная алгебра: \hfill \normalsize \date}

\vspace{5 pt}
{\bf \large \hfill  лекция 2\hfill }

\vspace{15 pt}
\centerline{ \large  Домашнее задание.}
\centerline{ \large  Кирилл Сетдеков}



\vspace*{10pt}
\setcounter{z}{0}

}

\begin{document}
\HSEhat


\subsection*{Задачи}

\begin{enumerate}

\item  Найдите определитель матрицы

$$A=
\begin{pmatrix}
1&2&3\\
5&1&4\\
3&2&5
\end{pmatrix}
$$

\vspace{5pt}
\textbf{Решение:}\\
Запишем как сумму алгебраических дополнений:
$$\det A = 1\cdot\det \begin{pmatrix}
1&4\\
2&5
\end{pmatrix}-5\cdot\det \begin{pmatrix}
2&3\\
2&5
\end{pmatrix}+3\cdot\det \begin{pmatrix}
2&3\\
1&4
\end{pmatrix} = -3-5\cdot(4)+3\cdot5=-8$$

\textbf{Ответ: $-8$}


\item  Найдите характеристический многочлен и спектр матрицы 

$$B=
\begin{pmatrix}
1 & 2 & -3\\
-5 & 1 & -4\\
0 & -2 & 4
\end{pmatrix}
$$

\vspace{5pt}
\textbf{Решение:}\\
Запишем, что мы хотим найти, характеристический многочлен:
$$\det(\lambda E - B)= \det \begin{pmatrix}
\lambda - 1 & -2 & 3\\
5 & \lambda -1 & 4\\
0 & 2 & \lambda - 4
\end{pmatrix}=$$
Разложим по 1 столбцу:
$$=(\lambda - 1)\det\begin{pmatrix}
 \lambda -1 & 4\\
 2 & \lambda - 4
\end{pmatrix} - 5 \det \begin{pmatrix}
 -2 & 3\\
 2 & \lambda - 4
\end{pmatrix} =$$

$=(\lambda-1)((\lambda-1)(\lambda-4)-8)+5(2(\lambda-4)+6)=(\lambda-1)(\lambda^2+4-5\lambda-8)+5(2\lambda-8+6)=(\lambda-1)(\lambda^2-5\lambda-4)+10(\lambda-1)=(\lambda-1)(\lambda^2-5\lambda-4+10)=(\lambda-1)(\lambda^2-5\lambda+6)=(\lambda-1)(\lambda-2)(\lambda-3)$

Спектр - решения характеристического многочлена равного 0.

$$(\lambda-1)(\lambda-2)(\lambda-3)=0 \Rightarrow \begin{cases}
    \lambda = 1\\
    \lambda = 2\\
    \lambda = 3
\end{cases}$$

\textbf{Ответ: Характеристический многочлен: $(\lambda-1)(\lambda-2)(\lambda-3)$ и спектр $\begin{cases}
    \lambda = 1\\
    \lambda = 2\\
    \lambda = 3
\end{cases}$}


\item Разлагая по второму столбцу, вычислите определитель

$$
\det\begin{pmatrix}
5&a&2&-1\\
4&b&4&-3\\
2&c&3&-2\\
4&d&5&-4
\end{pmatrix}
$$


\vspace{5pt}
\textbf{Решение:}\\
Запишем разложение по 2 столбцу:
$\det = -a \det \begin{pmatrix}
4&4&-3\\
2&3&-2\\
4&5&-4
\end{pmatrix}+b \det \begin{pmatrix}
5&2&-1\\
2&3&-2\\
4&5&-4
\end{pmatrix} - c \det \begin{pmatrix}
5&2&-1\\
4&4&-3\\
4&5&-4
\end{pmatrix}+ d \det \begin{pmatrix}
5&2&-1\\
4&4&-3\\
2&3&-2
\end{pmatrix}=\text{<разложим каждый определитель по 1 столбцу>}=-a(4\cdot(-12+10)-2\cdot(-16+15)+4(-8+9))+b(5\cdot(-12+10)-2\cdot(-8+5)+4\cdot(-4+3))-c(5\cdot(-16+15)-4\cdot(-8+5)+4\cdot(-6+4))+d(5\cdot(-8+9)-4\cdot(-4+3)+2\cdot(-6+4))=2a-8b+1c+5d$

\textbf{Ответ: Определитель равен: $2a-8b+c+5d$}
\item Найдите определитель. Запишите его в виде многочлена от $t$

$$
\det\begin{pmatrix}
-t&0&0&0&a_1\\
a_2&-t&0&0&0\\
0&a_3&-t&0&0\\
0&0&a_4&-t&0\\
0&0&0&a_5&-t
\end{pmatrix}
$$

\vspace{5pt}
\textbf{Решение:}\\
Разложим определитель по 1 колонке:
$=-t \det \begin{pmatrix}
-t&0&0&0\\
a_3&-t&0&0\\
0&a_4&-t&0\\
0&0&a_5&-t
\end{pmatrix} - a_2 \det \begin{pmatrix}
0&0&0&a_1\\
a_3&-t&0&0\\
0&a_4&-t&0\\
0&0&a_5&-t
\end{pmatrix}$ 
В первом определители - нижнетреугольная матрица, поэтому он равен произведению элементов на диагонали. Матрицу под правым определителем мы за 3 перестановки пар столбцов приведем к почти диагональному виду. От этого знак определителя изменится три раза.

$=-t \cdot t^4 + a_2 \det \begin{pmatrix}
a_1&0&0&0\\
0&a_3&-t&0\\
0&0&a_4&-t\\
-t&0&0&a_5
\end{pmatrix}=$

Разложим вторую матрицу по 1 колонке:

$=-t^5+a_2(a_1 \det \begin{pmatrix}
a_3&-t&0\\
0&a_4&-t\\
0&0&a_5
\end{pmatrix} +t \cdot \det \begin{pmatrix}
0&0&0\\
a_3&-t&0\\
0&a_4&-t
\end{pmatrix})=$ 

Получили нулевую строку $\Rightarrow$ правый определитель равен нулю, а слева - диагональная матрица.
Раскроем и получим ответ.
$=-t^5+a_2 \cdot a_1 \cdot a_3 \cdot a_4 \cdot a_5$


\textbf{Ответ: Определитель равен: $-t^5+a_1 \cdot a_2 \cdot a_3 \cdot a_4 \cdot a_5$}


\item Чему равен определитель матрицы, у которой сумма всех строк с четными номерами равна сумме всех строк с
нечетными номерами?

\vspace{5pt}

\textbf{Решение:}\\
Примем значение, которому равна сумма нечетных строк за $N$. Мы знаем, что $\det$ не меняется от элементарного преобразования типа "прибавим строку умноженную на коэффициент к другой строке". Соберем таким образом сумму всех нечетных строк в 1 строке и всех четных строк в строке 2.

Теперь мы можем получить строку, где все элементы нулевые, вычтя из 1 строки 2 (так как по условию сумма их равны). Таким образом, мы получили строку, где все элементы равны 0, при этом это преобразование не изменило $\det$. Определитель матрицы, которая имеет нулевую строку, равен 0.

\textbf{Ответ:  $0$}


\item Дана матрица 

$$
A=\begin{pmatrix}
1&1&0\\
0&1&0\\
0&3&3
\end{pmatrix}
$$


а) Найдите матрицу $A^{-1}$ с помощью элементарных преобразований.

\textbf{Решение:}\\
Запишем матрицу вместе с единичной
$
\left(\begin{array}{ccc|ccc}  
 1&1&0&1&0&0\\
0&1&0&0&1&0\\
0&3&3&0&0&1
\end{array}\right) \rightarrow$ вычтем \rom{1}-\rom{2} \\
$
\left(\begin{array}{ccc|ccc}  
 1&-1&0&1&-1&0\\
0&1&0&0&1&0\\
0&3&3&0&0&1
\end{array}\right) \rightarrow$ вычтем \rom{3}-3\rom{2} $
\left(\begin{array}{ccc|ccc}  
1&-1&0&1&-1&0\\
0&1&0&0&1&0\\
0&0&3&0&-3&1
\end{array}\right) \rightarrow$ \rom{3} разделить на 3 $
\left(\begin{array}{ccc|ccc}  
1&-1&0&1&-1&0\\
0&1&0&0&1&0\\
0&0&1&0&-1&\frac{1}{3}
\end{array}\right)$

\textbf{Ответ: 
$\begin{pmatrix}
1&-1&0\\
0&1&0\\
0&-1&\frac{1}{3}
\end{pmatrix}$}

б) Найдите матрицу $A^{-1}$ с помощью явной формулы (через присоединенную матрицу).

\textbf{Решение:}\\
Найдем определить А через разложение по столбцу 1.
$\det A = 1  \cdot (3-0)=3$

Найдем матрицу алгебраических дополнений.
$$\begin{pmatrix}
1\begin{vmatrix}
1 & 0  \\ 
3 & 3 
\end{vmatrix}&-1\begin{vmatrix}
0 & 0  \\ 
0 & 3 
\end{vmatrix}&1\begin{vmatrix}
0 & 1  \\ 
0 & 3 
\end{vmatrix}\\
-1\begin{vmatrix}
1 & 0  \\ 
3 & 3 
\end{vmatrix}&1\begin{vmatrix}
1 & 0  \\ 
0 & 3 
\end{vmatrix}&-1\begin{vmatrix}
1 & 1  \\ 
0 & 3 
\end{vmatrix}\\
1\begin{vmatrix}
1 & 0  \\ 
1 & 0 
\end{vmatrix}&-1\begin{vmatrix}
1 & 0  \\ 
0 & 0 
\end{vmatrix}&1\begin{vmatrix}
1 & 1  \\ 
0 & 1 
\end{vmatrix}
\end{pmatrix}=\begin{pmatrix}
3&0&0\\
-3&3&-3\\
0&0&-1
\end{pmatrix}=B$$

$$B^T=\begin{pmatrix}
3&-3&0\\
0&3&0\\
0&-3&-1
\end{pmatrix}$$

Обратная матрица будет равна транспонированной матрице алгебраических дополнений, деленной на определитель исходной матрицы.
$$\frac{B^T}{\det A}= \frac{1}{3}\begin{pmatrix}
3&-3&0\\
0&3&0\\
0&-3&-1
\end{pmatrix}= \begin{pmatrix}
1&-1&0\\
0&1&0\\
0&-1&\frac{1}{3}
\end{pmatrix}$$


\textbf{Ответ: 
$\begin{pmatrix}
1&-1&0\\
0&1&0\\
0&-1&\frac{1}{3}
\end{pmatrix}$}

\item
Найдите матрицу $X$, удовлетворяющую равенству

$$
X
\begin{pmatrix}
1&1&1\\
1&2&3\\
1&4&9
\end{pmatrix}
=
\begin{pmatrix}
1&2&3\\
2&4&6\\
3&6&9
\end{pmatrix}
$$

\textbf{Решение:}\\
Нам дано матричное уравнение $XA=B$. Домножим справа обе части на обратную матрицу к А.
$$XAA^{-1} = BA^{-1}$$
$$X = BA^{-1}$$

Если обратная матрица к A существует.

Запишем матрицу вместе с единичной
$
\left(\begin{array}{ccc|ccc}  
1&1&1&1&0&0\\
1&2&3&0&1&0\\
1&4&9&0&0&1
\end{array}\right) \rightarrow$ вычтем \rom{2}-\rom{1} и  \rom{3}-\rom{1} $
\left(\begin{array}{ccc|ccc}  
1&1&1&1&0&0\\
0&1&2&-1&1&0\\
0&3&8&-1&0&1
\end{array}\right) \rightarrow$ вычтем \rom{1}-\rom{2} и  \rom{3}-2\rom{2} $
\left(\begin{array}{ccc|ccc}  
1&0&-1&2&-1&0\\
0&1&2&-1&1&0\\
0&0&2&2&-3&1
\end{array}\right) \rightarrow$ вычтем \rom{2}-\rom{3} $
\left(\begin{array}{ccc|ccc}  
1&0&-1&2&-1&0\\
0&1&0&-3&4&-1\\
0&0&2&2&-3&1
\end{array}\right) \rightarrow$  прибавим \rom{1}+0,5\rom{3} и поделим \rom{3} на 2 \\$
\left(\begin{array}{ccc|ccc}  
1&0&0&3&-\frac{5}{2}&\frac{1}{2}\\
0&1&0&-3&4&-1\\
0&0&0&1&-\frac{3}{2}&\frac{1}{2}
\end{array}\right)$ 
$$A^{-1}=\begin{pmatrix}
3&-\frac{5}{2}&\frac{1}{2}\\
-3&4&-1\\
1&-\frac{3}{2}&\frac{1}{2}
\end{pmatrix}$$

$$BA^{-1}=\begin{pmatrix}
1&2&3\\
2&4&6\\
3&6&9
\end{pmatrix}  \begin{pmatrix}
3&-\frac{5}{2}&\frac{1}{2}\\
-3&4&-1\\
1&-\frac{3}{2}&\frac{1}{2}
\end{pmatrix}=\begin{pmatrix}
0&1&0\\
0&2&0\\
0&3&0
\end{pmatrix}$$


\textbf{Ответ: 
$X=\begin{pmatrix}
0&1&0\\
0&2&0\\
0&3&0
\end{pmatrix}$}

\end{enumerate}
\end{document}