\documentclass[a4paper,12pt]{article}
\usepackage[utf8]{inputenc}
\usepackage[cm,empty]{fullpage}
\usepackage[T2A]{fontenc}
\usepackage[english, russian]{babel}
\usepackage{amssymb,amsmath,amsxtra,amsthm}
\usepackage{proof}
\usepackage[pdftex]{graphicx}
\usepackage{wrapfig}
\usepackage{braket}
\usepackage{xcolor}

\usepackage[left=2cm,right=2cm,
    top=1cm,bottom=1cm,bindingoffset=0cm]{geometry}

\renewcommand{\leq}{\leqslant}
\renewcommand{\geq}{\geqslant}


\newcommand{\iiff}{\Longleftrightarrow}
\renewcommand{\iff}{\Leftrightarrow}
\newcommand{\nothing}{\varnothing}

\newtheorem*{rem}{Замечание}

\newcommand{\NN}{\mathbb{N}}
\newcommand{\ZZ}{\mathbb{Z}}
\newcommand{\Q}{\mathbb{Q}}
\newcommand{\A}{\mathbb{A}}
\newcommand{\R}{\mathbb{R}}
\renewcommand{\C}{\mathbb{C}}

\renewcommand{\phi}{\varphi}
\newcommand{\eps}{\varepsilon}

\makeatletter
\newcommand*{\rom}[1]{\expandafter\@slowromancap\romannumeral #1@}
\makeatother

\newcounter{z}


\newcommand{\zs}{\refstepcounter{z}\vskip 10pt\par\noindent
\fbox{\textbf{12.\arabic{z}}} }

\newcommand{\z}{\refstepcounter{z}\vskip 20pt\noindent
\fbox{\textbf{\arabic{z}}} }

\renewcommand{\date}{{\bf 28 марта 2021}} 

\newcommand{\dif}
{
------------------------------------------------------------------------------------------------------------------------------------------------------
}

\newcommand{\HSEhat}{
\vspace*{-0pt}
\noindent
\setcounter{z}{0}


{\bf \phantom{\date}  \large \hfill Линейная алгебра: \hfill \normalsize \date}

\vspace{5 pt}
{\bf \large \hfill  лекция 3\hfill }

\vspace{15 pt}
\centerline{ \large  Домашнее задание.}
\centerline{ \large  Кирилл Сетдеков}



\vspace*{10pt}
\setcounter{z}{0}

}

\begin{document}
\HSEhat


\subsection*{Задачи}

\begin{enumerate}
\item Решите квадратное уравнение над $\mathbb C$:
\[
-\frac{z^2}i + \frac{20z}{i - 3} + (2 - 11i) = 0.
\]

\textbf{Решение:}\\
Найдем определитель:
$$D=b^2-4ac =  (\frac{20}{i - 3})^2+4\frac{2 - 11i}i=\frac{400}{i^2+9-6i}+\frac{8-44i}{i}=\frac{400}{8-6i}+\frac{8-44i}{i}=$$ $$=\frac{400i+64-264-48i-352i}{8i+6}=\frac{-200}{8i+6}=-100 (\frac{1}{4i+3})=-100(\frac{3}{25}-\frac{4i}{25})=-12+16i=D$$

$$\sqrt{D}=\sqrt{4}\sqrt{-3+4i}=\sqrt{4}(\sqrt{\frac{\sqrt{25}-3}{2}}+i\sqrt{\frac{\sqrt{25}+3}{2}})=\sqrt{4}+4i=2+4i$$

Решение будет:
$$x_{1,2} = \frac{-b\pm (2+4i)}{2a}=\frac{\frac{20}{3-i}\pm (2+4i)}{\frac{2}{-i}}=-10\frac{i}{3-i}\pm i (1+2i)=-10\frac{i}{3-i}\pm(i-2) =$$
$$= -10i(\frac{3}{10}+\frac{i}{10})\pm(i-2)=-3i+1 \pm(i-2)$$
После того как возьмем два разных знака, то получим корни:
$x_1 = 3-4i; x_2=-1-2i$


\textbf{Ответ: $x_1 = 3-4i; x_2=-1-2i$}


\item Найдите все значения $\lambda\in \mathbb R$, при которых вектор $v$ линейно выражается через векторы $a_1$, $a_2$, $a_3$, где
\[
a_1 = 
\begin{pmatrix}
{2}\\{3}\\{5}
\end{pmatrix},\;
a_2 = 
\begin{pmatrix}
{3}\\{7}\\{8}
\end{pmatrix},\;
a_3 = 
\begin{pmatrix}
{1}\\{-6}\\{1}
\end{pmatrix},\;
v = 
\begin{pmatrix}
{7}\\{-2}\\{\lambda}
\end{pmatrix}.
\]

\vspace{5pt}
\textbf{Решение:}\\
Запишем вектора как столбцы матрицы и попробуем методом Гаусса прийти к ступенчатому виду, чтобы найти главные вектора и выразить через них зависимые.

$
\left(\begin{array}{cccc}  
2&3&1&7\\
3&7&-6&-2\\
5&8&1&\lambda
\end{array}\right) \rightarrow$ поменяем местами \rom{1} и \rom{2} \\
$
\left(\begin{array}{cccc}  
3&7&-6&-2\\
2&3&1&7\\
5&8&1&\lambda
\end{array}\right) \rightarrow$ вычтем  \rom{1} - \rom{2} \\
$
\left(\begin{array}{cccc}  
1&4&-7&-9\\
2&3&1&7\\
5&8&1&\lambda
\end{array}\right) \rightarrow$ вычтем  \rom{2} -2 \rom{1} и  \rom{3} -5 \rom{1}\\
$
\left(\begin{array}{cccc}  
1&4&-7&-9\\
0&-5&15&25\\
0&-12&36&\lambda+45
\end{array}\right) \rightarrow$ поделим  \rom{2} на -5\\
$
\left(\begin{array}{cccc}  
1&4&-7&-9\\
0&1&-3&-5\\
0&-12&36&\lambda+45
\end{array}\right) \rightarrow$ вычтем  \rom{1} -4 \rom{2}  и \rom{3} +12\rom{2}\\
$
\left(\begin{array}{cccc}  
1&0&5&11\\
0&1&-3&-5\\
0&0&0&\lambda-15
\end{array}\right)$

Если $\lambda \neq 15$, то $a_4$ - главный вектор, если равно, то его можно выразить через 1 и 2 вектор. Запишем систему:

$$\begin{cases}
    a_1+5a_3+11a_4=0\\
    a_2-3a_3-5a_4=0
\end{cases} \Rightarrow \begin{cases}
    a_3 = \frac{1}{8}(5a_1+11a_2)\\
    a_4 = \frac{1}{8}(-3a_1-5a_2)
\end{cases}$$

\textbf{Ответ: $\lambda \neq 15$, тогда $a_4 = \frac{1}{8}(-3a_1-5a_2)$. Когда $\lambda = 15$, $a_4$ линейно независим от остальных и является одним из 3 главных векторов}

\item а) Пусть $V = M_2(\mathbb{R})$, $U = \{\left(\begin{smallmatrix}
a & b \\
c & d
\end{smallmatrix} \right) \mid a + d = 0\}$. Покажите, что $U$ -- подпространство $V$. Иными словами, нужно показать, что множество $2\times 2$ матриц с нулевым следом является подпространством пространства всех вещественных $2\times 2$ матриц ({\it следом} матрицы называется сумма элементов главной диагонали матрицы).
Найдите размерность $U$ и предъявите некоторый базис $U$.

\textbf{Решение:}\\
Пространство $V = M_2(\mathbb{R})$ имеет размерность $\dim V = 4$ пример матриц, которые образуют базис: $X = \left(\begin{smallmatrix}
1 & 0 \\
0 & 0
\end{smallmatrix} \right)$, $Y = \left(\begin{smallmatrix}
0 & 1 \\
0 & 0
\end{smallmatrix} \right)$, $Z = \left(\begin{smallmatrix}
0 & 0 \\
1 & 0
\end{smallmatrix} \right)$, $T = \left(\begin{smallmatrix}
0 & 0 \\
0 & 1
\end{smallmatrix} \right)$ и любая матрица 2X2 имеет координаты $v = 
\begin{pmatrix}
{a}\\{b}\\{c}\\{d}
\end{pmatrix}$ 

Так как $a+d=0$, то можно любую матрицу из U представить в координатах пространства V: $u = \begin{pmatrix}
{a}\\{b}\\{c}\\{-a}\end{pmatrix}$


\textbf{Ответ:
$\dim U = 3$ 
Пример базиса:
$X = \left(\begin{smallmatrix}
1 & 0 \\
0 & -1
\end{smallmatrix} \right)$, $Y = \left(\begin{smallmatrix}
0 & 1 \\
0 & 0
\end{smallmatrix} \right)$, $Z = \left(\begin{smallmatrix}
0 & 0 \\
1 & 0
\end{smallmatrix} \right)$}



б) В выбранном базисе найдите координаты матрицы $\left(\begin{smallmatrix}
-1 & 2 \\
3 & 1
\end{smallmatrix} \right)$ (то есть укажите, как эта матрица выражается через базисные).

\vspace{5pt}
\textbf{Решение:}\\
В базисе, указанном выше, эту матрицу можно получить как $-1 X+2Y+3Z= -1\left(\begin{smallmatrix}
1 & 0 \\
0 & -1
\end{smallmatrix} \right) +2 \left(\begin{smallmatrix}
0 & 1 \\
0 & 0
\end{smallmatrix} \right)+3 \left(\begin{smallmatrix}
0 & 0 \\
1 & 0
\end{smallmatrix} \right)= \left(\begin{smallmatrix}
-1 & 2 \\
3 & 1
\end{smallmatrix} \right)$


\textbf{Ответ:
$\begin{pmatrix}
{-1}\\{2}\\{3}\end{pmatrix}$ 
В базисе:
$X = \left(\begin{smallmatrix}
1 & 0 \\
0 & -1
\end{smallmatrix} \right)$, $Y = \left(\begin{smallmatrix}
0 & 1 \\
0 & 0
\end{smallmatrix} \right)$, $Z = \left(\begin{smallmatrix}
0 & 0 \\
1 & 0
\end{smallmatrix} \right)$}

\item Найдите какой-нибудь базис системы векторов и выразите через него все остальные векторы системы:
$$a_1 = (2, -1, 3, 5), \ a_2 = (4, -3, 1, 3), \ a_3 = (3, -2, 3, 4), \ a_4 = (4, -1, 15, 17), \ a_5 = (7, -6, -7, 0).$$ 

\vspace{5pt}
\textbf{Решение:}\\
Запишем вектора как столбцы в матрицу и начнем применять метод Гаусса. Таким образом найдем какие из векторов главные и выразим через них остальные.:

$
\left(\begin{array}{ccccc}  
2&4&3&4&7\\
-1&-3&-2&-1&-6\\
3&1&3&15&-7\\
5&3&4&17&0
\end{array}\right) \rightarrow$ прибавим \rom{2} к \rom{1} \\
$
\left(\begin{array}{ccccc}  
1&1&1&3&1\\
-1&-3&-2&-1&-6\\
3&1&3&15&-7\\
5&3&4&17&0
\end{array}\right) \rightarrow$ прибавим \rom{2} к \rom{1}, \rom{3} -3 \rom{1}, \rom{3} -5 \rom{1}\\
$
\left(\begin{array}{ccccc}  
1&1&1&3&1\\
0&-2&-1&2&-5\\
0&-2&0&6&-10\\
0&-2&-1&2&-5
\end{array}\right) \rightarrow$ вычтем \rom{2} - \rom{3}, \rom{4} -\rom{3}\\
$
\left(\begin{array}{ccccc}  
1&1&1&3&1\\
0&0&-1&-4&5\\
0&-2&0&6&-10\\
0&0&-1&-4&5
\end{array}\right) \rightarrow$ поделим \rom{3} на -2 и поменяем с \rom{2}\\
$
\left(\begin{array}{ccccc}  
1&1&1&3&1\\
0&1&0&-3&5\\
0&0&-1&-4&5\\
0&0&-1&-4&5
\end{array}\right) \rightarrow$  \rom{4} - \rom{3}\\
$
\left(\begin{array}{ccccc}  
1&1&1&3&1\\
0&1&0&-3&5\\
0&0&-1&-4&5\\
0&0&0&0&0
\end{array}\right) \rightarrow$  \rom{1} - \rom{2} + \rom{3} \\
$
\left(\begin{array}{ccccc}  
1&0&0&2&1\\
0&1&0&-3&5\\
0&0&-1&-4&5\\
0&0&0&0&0
\end{array}\right) \rightarrow$  \rom{3} *-1 \\
$
\left(\begin{array}{ccccc}  
1&0&0&2&1\\
0&1&0&-3&5\\
0&0&1&4&-5\\
0&0&0&0&0
\end{array}\right) $

Мы получили, что вектора $(a_1,a_2,a_3)$ главные, скажем что они базис. В этом базисе координаты $a_4:\begin{pmatrix}2\\-3\\4
\end{pmatrix}$, координаты $a_5:\begin{pmatrix}1\\5\\-5
\end{pmatrix}$

\textbf{Ответ: базис - вектора $(a_1,a_2,a_3)$.  В этом базисе координаты $a_4:\begin{pmatrix}2\\-3\\4
\end{pmatrix}$, координаты $a_5:\begin{pmatrix}1\\5\\-5
\end{pmatrix}$}

\item Найдите размерность и предъявите базис следующего подпространства в $\mathbb{R}^5$, заданного множеством решений следующей однородной системы линейных уравнений:

\[
\left\{
\begin{aligned}
& 2x_1 - x_2 + x_3 - 2x_4 + 4x_5 = 0\\
& 4x_1 - 2x_2 + 5x_3 + x_4 + 7x_5 = 0\\
& 2x_1 - x_2 + x_3 + 8x_4 + 2x_5 = 0
\end{aligned}
\right.
\]

\vspace{5pt}
\textbf{Решение:}\\
У нас матричное уравнение $Ax=0$. Запишем матрицу $A$ и приведем ее к каноническому ступенчатому виду:
$
\left(\begin{array}{ccccc}  
2&-1&1&-2&4\\
4&-2&5&1&7\\
2&-1&1&8&2
\end{array}\right) \rightarrow$ вычтем \rom{2} -2 \rom{1} и \rom{3} - \rom{1}\\
$
\left(\begin{array}{ccccc}  
2&-1&1&-2&4\\
0&0&3&5&-1\\
0&0&0&10&-2
\end{array}\right) \rightarrow$ поделим \rom{1}/2; \rom{2}/3 и \rom{3}/10\\
$
\left(\begin{array}{ccccc}  
1&-1/2&1/2&-1&2\\
0&0&1&5/3&-1/3\\
0&0&0&1&-1/5
\end{array}\right)$

Размерность - 3 так как мы получили 3 главных вектора.
Выразим систему уравнений для $x_2$ и $x_5$
$$\begin{cases}
    x_4-x_5/5=0\\
    x_1+x_2/2+x_3/2-x_4+2x_5=0
\end{cases} \Rightarrow \begin{cases}
    x_5=5x_4\\
    x_2 = 2x_1+x_3+18x_4
\end{cases}$$

Запишем ФСР для этой системы:

$v_1=\begin{pmatrix}1\\2\\0\\0\\0
\end{pmatrix}$,$v_3=\begin{pmatrix}0\\1\\1\\0\\0\end{pmatrix}$,$v_4=\begin{pmatrix}0\\18\\0\\1\\5\end{pmatrix}$


\textbf{Ответ: $\dim = 3$ базис - вектора: $v_1=\begin{pmatrix}1\\2\\0\\0\\0
\end{pmatrix}$,$v_3=\begin{pmatrix}0\\1\\1\\0\\0\end{pmatrix}$,$v_4=\begin{pmatrix}0\\18\\0\\1\\5\end{pmatrix}$}

\item Для каждого значения $\lambda\in \mathbb R$ найдите ранг матрицы
\[
\begin{pmatrix}
{-\lambda}&{1}&{2}&{3}&{1}\\
{1}&{-\lambda}&{3}&{2}&{1}\\
{2}&{3}&{-\lambda}&{1}&{1}\\
{3}&{2}&{1}&{-\lambda}&{1}\\
\end{pmatrix}.
\]

\textbf{Решение:}\\
Назовем матрицу $A$. Мы сразу можем сказать что 
$2 \leq rank A \leq 4$ так как в матрице $A$ - 4 строки и определитель углового минора из верхнего правого угла не равен 0 $\begin{vmatrix}
3 & 1  \\ 
2 & 1 
\end{vmatrix}=3-2=1 \neq 0$

Мы знаем, что ранг матрицы равен наивысшему из порядков всевозможных ненулевых миноров этой матрицы.

Предположим, что мы хотим найти условия, при которых ранг матрицы равен меньше 4. Тогда мы должны записать все миноры 4 порядка, найти их определители и найти такие $\lambda$, при которых эти определители равны 0.

Для нашей матрицы будет 5 миноров, полученных удаление одного из столбцов.

1) без 1 столбца:

$M_1 = \begin{pmatrix}
{1}&{2}&{3}&{1}\\
{-\lambda}&{3}&{2}&{1}\\
{3}&{-\lambda}&{1}&{1}\\
{2}&{1}&{-\lambda}&{1}\\
\end{pmatrix}$, $\det M_1 = \lambda(\lambda+2)(\lambda+4)=0$


2) без 2 столбца:

$M_2=\begin{pmatrix}
{-\lambda}&{2}&{3}&{1}\\
{1}&{3}&{2}&{1}\\
{2}&{-\lambda}&{1}&{1}\\
{3}&{1}&{-\lambda}&{1}\\
\end{pmatrix}$, $\det M_2 = -\lambda(\lambda+2)(\lambda+4)=0$

3) без 3 столбца:

$M_3 = \begin{pmatrix}
{-\lambda}&{1}&{3}&{1}\\
{1}&{-\lambda}&{2}&{1}\\
{2}&{3}&{1}&{1}\\
{3}&{2}&{-\lambda}&{1}\\
\end{pmatrix}$, $\det M_3 = \lambda(\lambda+2)(\lambda+4)=0$

4) без 4 столбца:

$M_4=\begin{pmatrix}
{-\lambda}&{1}&{2}&{1}\\
{1}&{-\lambda}&{3}&{1}\\
{2}&{3}&{-\lambda}&{1}\\
{3}&{2}&{1}&{1}\\
\end{pmatrix}$, $\det M_4 = -\lambda(\lambda+2)(\lambda+4)=0$


5) без 5 столбца:

$M_5= \begin{pmatrix}
{-\lambda}&{1}&{2}&{3}\\
{1}&{-\lambda}&{3}&{2}\\
{2}&{3}&{-\lambda}&{1}\\
{3}&{2}&{1}&{-\lambda}\\
\end{pmatrix}$, $\det M_5 = \lambda(\lambda+2)(\lambda+4)(\lambda-6)=0$

Все 5 определителей одновременно превращаются в 0, если $\lambda \in \{ 0,-2,-4\} $

Мы нашли 3 значения $\lambda$, когда $rank A < 4$, значит в остальных значениях, $rank A = 4$

Проверим минор 3 порядка:
$\begin{vmatrix}
{-\lambda}&{1}&{2}\\
{1}&{-\lambda}&{3}\\
{2}&{3}&{-\lambda}\\
\end{vmatrix}=-\lambda^3+14\lambda+12=0$

С помощью подстановки, проверим, являются ли 0, -2, -4 решением этого уравнения.

$$-0^3+14*0+12=12\neq0$$
$$2^3+14*-2+12=-8\neq0$$
$$4^3+14*-4+12=20\neq0$$

Следовательно определитель этого минора не равен 0, когда определители миноров 4 порядка равны нулю. Следовательно ранг матрицы будет равен 3, для значений $\lambda$, когда ранг не равен 4.


\textbf{Ответ: для $\lambda \in \{ 0,-2,-4\}$, ранг равен 3. Для всех остальных значений $\lambda \notin \{ 0,-2,-4\}$, ранг равен 4.}
\end{enumerate}
\end{document}