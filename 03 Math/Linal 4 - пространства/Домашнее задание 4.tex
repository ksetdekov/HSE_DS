\documentclass[a4paper]{article}
\usepackage[utf8]{inputenc}


\usepackage{amsfonts, amsthm, amssymb, amsmath}	
\usepackage{color}
\usepackage{graphicx}
\usepackage{mathrsfs}
\usepackage[russian]{babel}
\usepackage{ upgreek }
\usepackage{ stmaryrd }
\usepackage{multicol}
\usepackage[left=0.7cm,right=0.7cm,
top=0.7cm,bottom=0.7cm,bindingoffset=0cm]{geometry}

\theoremstyle{definition}
\newtheorem{Lem}{Лемма}[section]
\newtheorem{Th}[Lem]{Теорема}
\newtheorem{Def}[Lem]{Определение}
\newtheorem{Corollary}[Lem]{Следствие}
\newtheorem{State}[Lem]{Утверждение}
\newtheorem{Proposition}[Lem]{Предложение}
\theoremstyle{remark}
\newtheorem*{Note}{Замечание}
\newtheorem*{Decl}{Обозначение}
\newtheorem*{Agree}{Соглашение}
\theoremstyle{definition}
\newtheorem{Ex}[Lem]{Пример}
\newenvironment{Proof} {\par\noindent{\bf Доказательство. }} {\hfill$\square$}
\pagestyle{empty}

\newcommand{\sn}{\{1, 2,\dots, n\}}
\newcommand{\PathArrK}{\stackrel{k}\longrightarrow}
\newcommand{\PathArrL}{\stackrel{l}\longrightarrow}
\newcommand{\PathArr}[1]{\stackrel{#1}\longrightarrow}
\newcommand{\defEq}{\stackrel{\rm def}=}
\newcommand{\Ss}{\mathscr{S}}
\newcommand{\Mnb}{M_n(\mathbf{B})}
\newcommand{\MnS}{M_n(\mathscr{S})}
\newcommand{\BtoB}{\colon \Mnb \rightarrow \Mnb}
\newcommand{\StoS}{\colon \MnS \rightarrow \MnS}
\newcommand{\gper}{\mathcal{P}_{\lambda,n}}
\newcommand{\gperone}{\mathcal{P}_{\lambda-1,n-1}}
\newcommand{\GPart}[2]{(#1\!\shortrightarrow\! #2)}
\newcommand{\GPartC}{(G_1\!\shortrightarrow\!G_2)}
\newcommand{\mc}[1]{\mathcal{#1}}

\newcounter{zadacha}
%\newcommand{\z}{\par\textbf{Задача. }}
\newcommand{\z}{\par\noindent\addtocounter{zadacha}{1}%
	\textbf{\arabic{zadacha}.} }
\newcommand{\zh}{\par\noindent\addtocounter{zadacha}{1}%
	\textbf{\arabic{zadacha}*.} }

\begin{document}
\centerline{\bf{\Large Домашнее задание 4}}
\centerline{ {\it 31.03.2021}}
\vspace{10pt}


\subsection*{Общие сведения:}
\begin{itemize}

\item Квадратные матрицы $A, B\in M_{n}(\mathbb{R})$ называются сопряженными, если найдется невырожденная матрица $C\in M_{n}(\mathbb{R})$ такая, что $B = C^{-1}AC$. Сопряженные матрицы обладают похожими свойствами, в частности, имеют одинаковый определитель, след и ранг, а также одинаковые наборы собственных значений.

\item В пространстве $\mathbb R^n$ стандартным называется следующий базис
\[
e_1 = 
\begin{pmatrix}
{1}\\{0}\\{\vdots}\\{0}
\end{pmatrix},
e_2 = 
\begin{pmatrix}
{0}\\{1}\\{\vdots}\\{0}
\end{pmatrix},
\ldots,
e_n = 
\begin{pmatrix}
{0}\\{0}\\{\vdots}\\{1}
\end{pmatrix}
\]

\end{itemize}

\subsection*{Задачи:}




\z Какие из следующих матриц сопряжены? Если они сопряжены, то укажите, с помощью какой матрицы:
\begin{itemize}
\item[а)] $\begin{pmatrix}{1}&{0}\\{0}&{0}\end{pmatrix}$ и $\begin{pmatrix}{0}&{0}\\{0}&{1}\end{pmatrix};$
\item[б)] $\begin{pmatrix}{3}&{1}\\{1}&{2}\end{pmatrix}$ и $\begin{pmatrix}{3}&{-1}\\{2}&{1}\end{pmatrix};$
\item[в)] $\begin{pmatrix}{2}&{1}\\{1}&{1}\end{pmatrix}$ и $\begin{pmatrix}{2}&{1}\\{0}&{1}\end{pmatrix}$.
\end{itemize}

\vspace{5pt}


\z В пространстве $\mathbb R^3$ заданы следующие векторы:
\[
v_1 =
\begin{pmatrix}
{1}\\{0}\\{2}
\end{pmatrix}
,\;
v_2 =
\begin{pmatrix}
{1}\\{1}\\{-1}
\end{pmatrix}
,\;
v_3 =
\begin{pmatrix}
{2}\\{0}\\{3}
\end{pmatrix}
,\;
u_1 =
\begin{pmatrix}
{1}\\{0}\\{1}
\end{pmatrix}
,\;
u_2 =
\begin{pmatrix}
{0}\\{1}\\{1}
\end{pmatrix}
,\;
u_3 =
\begin{pmatrix}
{0}\\{0}\\{2}
\end{pmatrix}.
\]
Найдите матрицу $A$ линейного оператора $\varphi\colon \mathbb R^3\to \mathbb R^3$, заданного по правилу $x\mapsto Ax$, такого, что $A v_i = u_i$ при $i = 1, 2, 3$. 

\vspace{5pt}



\z Пусть $\varphi\colon \mathbb R^3\to \mathbb R^2$ -- линейное отображение, заданное в стандартном базисе матрицей $A = \left(\begin{smallmatrix}{1}&{2}&{0}\\{-1}&{0}&{2}\end{smallmatrix}\right)$.  Пусть
\[
f_1 = 
\begin{pmatrix}
{1}\\{1}\\{1}
\end{pmatrix},\:
f_2 = 
\begin{pmatrix}
{1}\\{1}\\{2}
\end{pmatrix},\:
f_3 = 
\begin{pmatrix}
{1}\\{2}\\{3}
\end{pmatrix}\text{ -- \ векторы в } \mathbb R^3,\quad
g_1 =
\begin{pmatrix}
{1}\\{2}
\end{pmatrix},\:
g_2 =
\begin{pmatrix}
{1}\\{1}
\end{pmatrix}\text{ -- \ векторы в } \mathbb R^3.
\]
Найти матрицу отображения $\varphi$ в базисах $f_1,f_2,f_3$ и $g_1,g_2$.

\vspace{5pt}



\z В пространстве $\mathbb R^3$ заданы следующие векторы:
\[
v_1 = 
\begin{pmatrix}
{1}\\{1}\\{1}
\end{pmatrix},\,
v_2 = 
\begin{pmatrix}
{1}\\{2}\\{0}
\end{pmatrix},\,
v_3 = 
\begin{pmatrix}
{5}\\{8}\\{2}
\end{pmatrix},\,
v_4 = 
\begin{pmatrix}
{-1}\\{-2}\\{1}
\end{pmatrix},\,
v_5 = 
\begin{pmatrix}
{3}\\{5}\\{-1}
\end{pmatrix}.
\]
Существует ли линейное отображение $\varphi\colon \mathbb R^3\to \mathbb R^2$ такое, что $\varphi(v_i) = u_i$ при $i = 1, 2, 3, 4, 5$, где

\[
u_1 = 
\begin{pmatrix}
{1}\\{0}
\end{pmatrix},\,
u_2 = 
\begin{pmatrix}
{0}\\{1}
\end{pmatrix},\,
u_3 = 
\begin{pmatrix}
{2}\\{3}
\end{pmatrix},\,
u_4 = 
\begin{pmatrix}
{0}\\{0}
\end{pmatrix},\,
u_5 = 
\begin{pmatrix}
{1}\\{0}
\end{pmatrix}?
\]

\vspace{5pt}

\z Найдите в стандартном базисе матрицу линейного оператора $\varphi\colon \mathbb R^2\to \mathbb R^2$, который сжимает плоскость в 2 раза вдоль прямой $y = -2x$ и одновременно растягивает плоскость в 3 раза вдоль прямой $y = x$.

\vspace{5pt}


\z Найти собственные значения и собственные векторы линейного оператора, заданного в стандартном базисе матрицей:



$$
\begin{pmatrix}
{4}&{-5}&{2}\\
{5}&{-7}&{3}\\
{6}&{-9}&{4}\\
\end{pmatrix}
$$

Можно ли эти матрицу диагонализовать в каком-нибудь базисе?

\vspace{5pt}


\z Пусть $a_1, a_2, \dots, a_n \in \mathbb{R}$. Найдите собственные значения $n \times n$ матрицы $x^T x$, где $x$ -- матрица-строка $(a_1, a_2,\ldots,a_n)$.


\end{document}