\documentclass[a4paper,12pt]{article}
\usepackage[utf8]{inputenc}
\usepackage[cm,empty]{fullpage}
\usepackage[T2A]{fontenc}
\usepackage[english, russian]{babel}
\usepackage{amssymb,amsmath,amsxtra,amsthm}
\usepackage{proof}
\usepackage[pdftex]{graphicx}
\usepackage{wrapfig}
\usepackage{braket}
\usepackage{xcolor}
\usepackage{enumitem}

\usepackage[left=2cm,right=2cm,
    top=1cm,bottom=1cm,bindingoffset=0cm]{geometry}

\renewcommand{\leq}{\leqslant}
\renewcommand{\geq}{\geqslant}


\newcommand{\iiff}{\Longleftrightarrow}
\renewcommand{\iff}{\Leftrightarrow}
\newcommand{\nothing}{\varnothing}

\newtheorem*{rem}{Замечание}

\newcommand{\NN}{\mathbb{N}}
\newcommand{\ZZ}{\mathbb{Z}}
\newcommand{\Q}{\mathbb{Q}}
\newcommand{\A}{\mathbb{A}}
\newcommand{\R}{\mathbb{R}}
\renewcommand{\C}{\mathbb{C}}

\renewcommand{\phi}{\varphi}
\newcommand{\eps}{\varepsilon}

\makeatletter
\newcommand*{\rom}[1]{\expandafter\@slowromancap\romannumeral #1@}
\makeatother

\newcounter{z}


\newcommand{\zs}{\refstepcounter{z}\vskip 10pt\par\noindent
\fbox{\textbf{12.\arabic{z}}} }

\newcommand{\z}{\refstepcounter{z}\vskip 20pt\noindent
\fbox{\textbf{\arabic{z}}} }

\renewcommand{\date}{{\bf 8 апреля 2021}} 

\newcommand{\dif}
{
------------------------------------------------------------------------------------------------------------------------------------------------------
}

\newcommand{\HSEhat}{
\vspace*{-0pt}
\noindent
\setcounter{z}{0}


{\bf \phantom{\date}  \large \hfill Линейная алгебра: \hfill \normalsize \date}

\vspace{5 pt}
{\bf \large \hfill  лекция 4\hfill }

\vspace{15 pt}
\centerline{ \large  Домашнее задание.}
\centerline{ \large  Кирилл Сетдеков}



\vspace*{10pt}
\setcounter{z}{0}

}

\begin{document}
\HSEhat





Квадратные матрицы $A, B\in M_{n}(\mathbb{R})$ называются сопряженными, если найдется невырожденная матрица $C\in M_{n}(\mathbb{R})$ такая, что $B = C^{-1}AC$. Сопряженные матрицы обладают похожими свойствами, в частности, имеют одинаковый определитель, след и ранг, а также одинаковые наборы собственных значений.

В пространстве $\mathbb R^n$ стандартным называется следующий базис
\[
e_1 = 
\begin{pmatrix}
{1}\\{0}\\{\vdots}\\{0}
\end{pmatrix},
e_2 = 
\begin{pmatrix}
{0}\\{1}\\{\vdots}\\{0}
\end{pmatrix},
\ldots,
e_n = 
\begin{pmatrix}
{0}\\{0}\\{\vdots}\\{1}
\end{pmatrix}
\]


\begin{enumerate}

\subsection*{Задачи:}




\item Какие из следующих матриц сопряжены? Если они сопряжены, то укажите, с помощью какой матрицы:
\begin{itemize}
\item[а)] $\begin{pmatrix}{1}&{0}\\{0}&{0}\end{pmatrix}$ и $\begin{pmatrix}{0}&{0}\\{0}&{1}\end{pmatrix};$

\textbf{Решение:}\\
Из условия выше следует, что можно проверить сначала определитель, след и ранг матриц на равенство, а потом только считать более сложное решение для $C$.
$$tr A = tr B = 1$$ 
$$\det A = \det B = 0$$
$$ rank A = rank B = 1$$
Посчитаем решение.

Нас интересует матрица $C=\begin{pmatrix}{a}&{b}\\{c}&{d}\end{pmatrix}$
$$C^{-1}=\frac{1}{ad-bc}\begin{pmatrix}{d}&{-b}\\{-c}&{a}\end{pmatrix}$$

Само уравнение будет иметь вид:
$$\begin{pmatrix}{1}&{0}\\{0}&{0}\end{pmatrix}=\frac{1}{ad-bc}\begin{pmatrix}{d}&{-b}\\{-c}&{a}\end{pmatrix} \begin{pmatrix}{0}&{0}\\{0}&{1}\end{pmatrix} \begin{pmatrix}{a}&{b}\\{c}&{d}\end{pmatrix}$$
Перемножим справа 1 и 2 матрицы:
$$\begin{pmatrix}{1}&{0}\\{0}&{0}\end{pmatrix}=\frac{1}{ad-bc}\begin{pmatrix}{0}&{-b}\\{0}&{a}\end{pmatrix} \begin{pmatrix}{a}&{b}\\{c}&{d}\end{pmatrix}$$
Перемножим оставшиеся матрицы:
$$\begin{pmatrix}{1}&{0}\\{0}&{0}\end{pmatrix}=\frac{1}{ad-bc} \begin{pmatrix}{-bc}&{-bd}\\{ac}&{ad}\end{pmatrix}$$
Нам нужно решение 
$$\begin{cases}
\frac{-bc}{ad-bc}=1\\
-bd=0\\
ac = 0\\
ad =0
\end{cases}$$

Решением будет $\begin{cases}a = 0\\
d = 0\\
bc \neq 0
\end{cases}$


\textbf{Ответ: Эти матрицы сопряжены. Они сопряжены с помощью матрицы вида $\begin{pmatrix}{0}&{b}\\{c}&{0}\end{pmatrix}$, где $bc \neq 0$}


\item[б)] $\begin{pmatrix}{3}&{1}\\{1}&{2}\end{pmatrix}$ и $\begin{pmatrix}{3}&{-1}\\{2}&{1}\end{pmatrix};$

\textbf{Ответ: Эти матрицы не сопряжены так как их следы не равны: $tr A = 5 \neq tr B = 4 $}


\item[в)] $\begin{pmatrix}{2}&{1}\\{1}&{1}\end{pmatrix}$ и $\begin{pmatrix}{2}&{1}\\{0}&{1}\end{pmatrix}$.

\textbf{Ответ: Эти матрицы не сопряжены так как их определители не равны: $|A| = 1 \neq |B| = 2 $}


\end{itemize}

\vspace{5pt}


\item В пространстве $\mathbb R^3$ заданы следующие векторы:
\[
v_1 =
\begin{pmatrix}
{1}\\{0}\\{2}
\end{pmatrix}
,\;
v_2 =
\begin{pmatrix}
{1}\\{1}\\{-1}
\end{pmatrix}
,\;
v_3 =
\begin{pmatrix}
{2}\\{0}\\{3}
\end{pmatrix}
,\;
u_1 =
\begin{pmatrix}
{1}\\{0}\\{1}
\end{pmatrix}
,\;
u_2 =
\begin{pmatrix}
{0}\\{1}\\{1}
\end{pmatrix}
,\;
u_3 =
\begin{pmatrix}
{0}\\{0}\\{2}
\end{pmatrix}.
\]
Найдите матрицу $A$ линейного оператора $\varphi\colon \mathbb R^3\to \mathbb R^3$, заданного по правилу $x\mapsto Ax$, такого, что $A v_i = u_i$ при $i = 1, 2, 3$. 

\vspace{5pt}

\textbf{Решение:}\\
$AV=U$, где V - матрица из векторов $v$, а U - матрица из векторов $u$. Нам нужно найти $A=UV^{-1}$

Найдем сначала $V^{-1}$ методом Гаусса.

$
\left(\begin{array}{ccc|ccc}  
1&1&2&1&0&0\\
0&1&0&0&1&0\\
2&-1&3&0&0&1
\end{array}\right) \rightarrow$ вычтем \rom{3} -2 \rom{1} \\
$
\left(\begin{array}{ccc|ccc}  
1&1&2&1&0&0\\
0&1&0&0&1&0\\
0&-3&-1&-2&0&1
\end{array}\right) \rightarrow$ вычтем \rom{1} - \rom{2} и   \rom{3}+ \rom{2}\\
$
\left(\begin{array}{ccc|ccc}  
1&0&2&1&-1&0\\
0&1&0&0&1&0\\
0&0&-1&-2&3&1
\end{array}\right) \rightarrow$ прибавим \rom{1} + 2 \rom{3} и   умножим \rom{3} на -1\\
$
\left(\begin{array}{ccc|ccc}  
1&0&0&-3&5&2\\
0&1&0&0&1&0\\
0&0&1&2&-3&-1
\end{array}\right) $
Мы нашли в правой части матрицу $V^{-1}$.

Рассчитаем A через произведение.

$$UV^{-1}=\begin{pmatrix}
1&0&0\\
0&1&0\\
1&1&2
\end{pmatrix}\begin{pmatrix}
-3&5&2\\
0&1&0\\
2&-3&-1
\end{pmatrix}=\begin{pmatrix}
-3&5&2\\
0&1&0\\
1&0&0
\end{pmatrix}$$

\textbf{Ответ: матрица $A=\begin{pmatrix}
-3&5&2\\
0&1&0\\
1&0&0
\end{pmatrix}$}


\item Пусть $\varphi\colon \mathbb R^3\to \mathbb R^2$ -- линейное отображение, заданное в стандартном базисе матрицей $A = \left(\begin{smallmatrix}{1}&{2}&{0}\\{-1}&{0}&{2}\end{smallmatrix}\right)$.  Пусть
\[
f_1 = 
\begin{pmatrix}
{1}\\{1}\\{1}
\end{pmatrix},\:
f_2 = 
\begin{pmatrix}
{1}\\{1}\\{2}
\end{pmatrix},\:
f_3 = 
\begin{pmatrix}
{1}\\{2}\\{3}
\end{pmatrix}\text{ -- \ векторы в } \mathbb R^3,\quad
g_1 =
\begin{pmatrix}
{1}\\{2}
\end{pmatrix},\:
g_2 =
\begin{pmatrix}
{1}\\{1}
\end{pmatrix}\text{ -- \ векторы в } \mathbb R^3.
\]
Найти матрицу отображения $\varphi$ в базисах $f_1,f_2,f_3$ и $g_1,g_2$.

\vspace{5pt}

\textbf{Решение:}\\
Согласно теореме с лекции, при смене базисов, нужно матрицу линейного отображения умножить слева на обратную матрицу перехода к новому базису в целевом пространстве и справа на матрицу перехода к новому базису в исходном пространстве: $A' = C^{-1}_{e\rightarrow g} A C_{e\rightarrow f}$

$A$ уже известно, $C_{e\rightarrow f}$ - это матрица их колонок векторов $f_1,f_2,f_3$.
$$C^{-1}_{e\rightarrow g} = inv \begin{pmatrix}
1&1\\
2&1
\end{pmatrix} =\frac{1}{-1}\begin{pmatrix}
1&-1\\
-2&1
\end{pmatrix}=\begin{pmatrix}
-1&1\\
2&-1
\end{pmatrix}  $$

Завершим расчет:
$$A'=\begin{pmatrix}
-1&1\\
2&-1
\end{pmatrix} \begin{pmatrix}{1}&{2}&{0}\\{-1}&{0}&{2}\end{pmatrix} \begin{pmatrix}
1&1&1\\
1&1&2\\
1&2&3
\end{pmatrix}=\begin{pmatrix}-2&-2&2\\
3&4&-2\\\end{pmatrix} \begin{pmatrix}
1&1&1\\
1&1&2\\
1&2&3
\end{pmatrix}=\begin{pmatrix}
-2&0&0\\
5&3&5\\
\end{pmatrix}$$

\textbf{Ответ: матрица  отображения $\varphi$ в базисах $f_1,f_2,f_3$ и $g_1,g_2$ $A'=\begin{pmatrix}
-2&0&0\\
5&3&5\\
\end{pmatrix}$}

\item В пространстве $\mathbb R^3$ заданы следующие векторы:
\[
v_1 = 
\begin{pmatrix}
{1}\\{1}\\{1}
\end{pmatrix},\,
v_2 = 
\begin{pmatrix}
{1}\\{2}\\{0}
\end{pmatrix},\,
v_3 = 
\begin{pmatrix}
{5}\\{8}\\{2}
\end{pmatrix},\,
v_4 = 
\begin{pmatrix}
{-1}\\{-2}\\{1}
\end{pmatrix},\,
v_5 = 
\begin{pmatrix}
{3}\\{5}\\{-1}
\end{pmatrix}.
\]
Существует ли линейное отображение $\varphi\colon \mathbb R^3\to \mathbb R^2$ такое, что $\varphi(v_i) = u_i$ при $i = 1, 2, 3, 4, 5$, где

\[
u_1 = 
\begin{pmatrix}
{1}\\{0}
\end{pmatrix},\,
u_2 = 
\begin{pmatrix}
{0}\\{1}
\end{pmatrix},\,
u_3 = 
\begin{pmatrix}
{2}\\{3}
\end{pmatrix},\,
u_4 = 
\begin{pmatrix}
{0}\\{0}
\end{pmatrix},\,
u_5 = 
\begin{pmatrix}
{1}\\{0}
\end{pmatrix}?
\]

\vspace{5pt}

\textbf{Решение:}\\
Переход между векторами $V\rightarrow U$ будет соответствовать умножению V на некоторую матрицу $A_{2*3}$.

$$U = AV$$
$$A=UV^{-1}$$
При этом в матрице $V$ будут векторы, которые образуют базис их 5 векторов $v_1,...,v_5$, а в матрице U - соответствующие им вектора из $u_1,...,u_5$
Найдем базис пространства из векторов $v$

Привели к ступенчатому виду матрицу их всех векторов V и сразу найдем базисные вектора и обратную матрицу к ним.

$
\left(\begin{array}{ccccc|ccc}  
1&1&5&-1&3&1&0&0\\
1&2&8&-2&5&0&1&0\\
1&0&2&1&-1&0&0&1\\
\end{array}\right) \rightarrow$ вычтем \rom{2}-\rom{1} и \rom{3}-\rom{1}  \\
$
\left(\begin{array}{ccccc|ccc}  
1&1&5&-1&3&1&0&0\\
0&1&3&-1&2&-1&1&0\\
0&-1&-3&2&-4&-1&0&1\\
\end{array}\right) \rightarrow$ вычтем \rom{1}-\rom{2} и \rom{3}+\rom{2}  \\
$
\left(\begin{array}{ccccc|ccc}  
1&0&2&0&1&2&-1&0\\
0&1&3&-1&2&-1&1&0\\
0&0&0&1&-2&-2&1&1\\
\end{array}\right) \rightarrow$  \rom{2}+\rom{3}\\
$
\left(\begin{array}{ccccc|ccc}  
1&0&2&0&1&2&-1&0\\
0&1&3&0&0&-3&2&1\\
0&0&0&1&-2&-2&1&1\\
\end{array}\right)$ 


Мы получили, что базис образован векторами $v_1, v_2, v_4$.
Матрица $V=\begin{pmatrix}
1&1&-1\\
1&2&-2\\
1&0&1&\\
\end{pmatrix}$ и обратная ей: $V^{-1} = \begin{pmatrix}
2&-1&0\\
-3&2&1\\
-2&1&1\\
\end{pmatrix}$

Для пространства из векторов $u$, можно сразу заметить, что есть вектор $u_5$ и $u_2$ - и это стандартный базис


Составим $U = \begin{pmatrix}
v_1|& v_2|& v_4
\end{pmatrix}$
и умножим ее на $V^{-1}$

$$A=UV^{-1}=\begin{pmatrix}
1&0&0\\
0&1&0
\end{pmatrix}
\begin{pmatrix}
2&-1&0\\
-3&2&1\\
-2&1&1\\
\end{pmatrix}=\begin{pmatrix}
2&-1&0\\
-3&2&1
\end{pmatrix}$$
 
По построению, векторы 1, 2, 4 переходят их $v$ в $u$. Для векторов 3 и 5 также выполняется это преобразование, что можно проверить.

$$Av_3 =\begin{pmatrix}
2&-1&0\\
-3&2&1
\end{pmatrix}\begin{pmatrix}
{5}\\{8}\\{2}
\end{pmatrix}=\begin{pmatrix}
{2}\\{3}
\end{pmatrix}=u_3$$

$$Av_5 =\begin{pmatrix}
2&-1&0\\
-3&2&1
\end{pmatrix}\begin{pmatrix}
{3}\\{5}\\{-1}
\end{pmatrix}=\begin{pmatrix}
{1}\\{0}
\end{pmatrix}=u_5$$


\textbf{Ответ: Да, такое линейное отображение $\varphi$ существует и его матрица $\begin{pmatrix}
2&-1&0\\
-3&2&1
\end{pmatrix}$}

\item Найдите в стандартном базисе матрицу линейного оператора $\varphi\colon \mathbb R^2\to \mathbb R^2$, который сжимает плоскость в 2 раза вдоль прямой $y = -2x$ и одновременно растягивает плоскость в 3 раза вдоль прямой $y = x$.

\vspace{5pt}


\textbf{Решение:}\\
Положим мы  уже находимся в базисе, где происходит эта трансформация, тогда трансформации соответствует матрице оператора  $A = \begin{pmatrix}{3}&{0}\\{0}&{0,5}\end{pmatrix}$
Найдем матрицу перехода от базиса в векторах $\begin{pmatrix}
{1}\\{1}
\end{pmatrix}$ и $\begin{pmatrix}
{1}\\{-2}
\end{pmatrix}$ к стандартному. Сделаем это методом гаусса.

$
\left(\begin{array}{cc|cc}  
1&1&1&0\\
1&-2&0&1
\end{array}\right) \rightarrow$ вычтем \rom{2}-\rom{1} \\
$
\left(\begin{array}{cc|cc}  
1&1&1&0\\
0&-3&-1&1
\end{array}\right) \rightarrow$ поделим \rom{2} на -3 \\
$
\left(\begin{array}{cc|cc}  
1&1&1&0\\
0&1&\frac{1}{3}&-\frac{1}{3}
\end{array}\right) \rightarrow$ вычтем \rom{1}-\rom{2} \\
$
\left(\begin{array}{cc|cc}  
1&0&\frac{2}{3}&\frac{1}{3}\\
0&1&\frac{1}{3}&-\frac{1}{3}
\end{array}\right)$

Матрица перехода к стандартному базису - $T = \begin{pmatrix}
\frac{2}{3}&\frac{1}{3}\\
\frac{1}{3}&-\frac{1}{3}
\end{pmatrix}$

Матрицу $T^{-1}$ мы уже знаем - это записанные по столбцам координаты векторов, относительно которых происходит растяжение и сжатие. $T^{-1} = \begin{pmatrix}
1&1\\1&-2 
\end{pmatrix}$

Наш ответ относительно матрицы $A^*$ линейного оператора $\varphi$:

$$A^* = T^{-1} A T= \begin{pmatrix}
1&1\\1&-2
\end{pmatrix} \begin{pmatrix}{3}&{0}\\{0}&{0,5}\end{pmatrix} \begin{pmatrix}
\frac{2}{3}&\frac{1}{3}\\
\frac{1}{3}&-\frac{1}{3}
\end{pmatrix} = \begin{pmatrix}
3&0.5\\3&-1
\end{pmatrix}  \begin{pmatrix}
\frac{2}{3}&\frac{1}{3}\\
\frac{1}{3}&-\frac{1}{3}
\end{pmatrix} = \frac{1}{6}\begin{pmatrix}
13&5\\
10&8
\end{pmatrix}=\begin{pmatrix}
2\frac{1}{6}&\frac{5}{6}\\
1\frac{2}{3}&1\frac{1}{3}
\end{pmatrix}$$

\textbf{Ответ: в стандартном базисе матрица линейного оператора $\begin{pmatrix}
2\frac{1}{6}&\frac{5}{6}\\
1\frac{2}{3}&1\frac{1}{3}
\end{pmatrix}$}

\item Найти собственные значения и собственные векторы линейного оператора, заданного в стандартном базисе матрицей:



$$
\begin{pmatrix}
{4}&{-5}&{2}\\
{5}&{-7}&{3}\\
{6}&{-9}&{4}\\
\end{pmatrix}
$$

Можно ли эти матрицу диагонализовать в каком-нибудь базисе?

\vspace{5pt}
\textbf{Решение:}\\

Найдем собственные значения, решив уравнение относительно $\lambda$:

$$\begin{vmatrix} {4-\lambda}&{-5}&{2}\\
{5}&{-7-\lambda}&{3}\\
{6}&{-9}&{4-\lambda}\end{vmatrix} =(4-\lambda)\begin{vmatrix}
{-7-\lambda}&{3}\\
{-9}&{4-\lambda}
\end{vmatrix}-5\begin{vmatrix}
{-5}&{2}\\
{-9}&{4-\lambda}
\end{vmatrix}+6\begin{vmatrix}
{-5}&{2}\\
{-7-\lambda}&{3}
\end{vmatrix}=$$
$=(4-\lambda)(-28+7\lambda-4\lambda+\lambda+27)-5(-20+5\lambda+18)+6(-15+14+2\lambda)=(4-\lambda)(4\lambda-1)-5(5\lambda-2)+6(2\lambda-1)=16\lambda-4-4\lambda^2+\lambda-25\lambda+10+12\lambda-6=-4\lambda^2+4\lambda=-4\lambda(1-\lambda)=0$

Корни $\lambda=0$ и $\lambda=1$ - это собственные значения.

Про диагонализируемость.\\
Если у матрицы n×n ровно n различных собственных значений, то она диагонализируема. Так как у нашей матрицы только 2 различных собственных значения, то она не диагонализируема. Собственные значения не меняются при переходе в другой базис $\Rightarrow$  эту матрицу нельзя диагонализовать в любом базисе. 

\textbf{Ответ: собственные значения: 1 и 0. Эту матрицу нельзя диагонализовать в любом базисе.}


\item Пусть $a_1, a_2, \dots, a_n \in \mathbb{R}$. Найдите собственные значения $n \times n$ матрицы $x^T x$, где $x$ -- матрица-строка $(a_1, a_2,\ldots,a_n)$.

\textbf{Решение:}\\
Для начала найдем собственные значения для первых $n$ в этой последовательности увеличивающихся матриц.

$n=1$: матрица имеет вид $
\begin{pmatrix}
a^2
\end{pmatrix}
$. Найдем собственные значения из уравнения $\begin{vmatrix}
a^2-\lambda
\end{vmatrix} =0$. собственные значения - только $a^2$

$n=2$: матрица имеет вид $
\begin{pmatrix}
a^2&ab\\
ab&b^2
\end{pmatrix}
$. Найдем собственные значения из уравнения $\begin{vmatrix}
a^2-\lambda&ab\\
ab&b^2-\lambda
\end{vmatrix} =a^2b^2-a^2\lambda-{\lambda} b^2 + \lambda^2 -a^2b^2=-a^2\lambda-{\lambda} b^2 + \lambda^2=\lambda(\lambda-(a^2+b^2))$. Собственные значения - 0 и $a^2+b^2$

$n=3$: матрица имеет вид $
\begin{pmatrix}
a^2&ab&ac\\
ab&b^2&bc\\
ac&bc&c^2
\end{pmatrix}
$. Найдем собственные значения из уравнения $\begin{vmatrix}
a^2-\lambda&ab&ac\\
ab&b^2-\lambda&bc\\
ac&bc&c^2-\lambda
\end{vmatrix} =(a^2-\lambda)\begin{vmatrix}
b^2-\lambda&bc\\
bc&c^2-\lambda
\end{vmatrix}-ab\begin{vmatrix}
ab&ac\\
bc&c^2-\lambda
\end{vmatrix}+ac\begin{vmatrix}
ab&ac\\
b^2-\lambda&bc
\end{vmatrix}=(a^2 + b^2 + c^2 - \lambda) \lambda^2=0$. Собственные значения - 0,0 и $a^2 + b^2 + c^2$

Также заметим, что начиная с $n=2$, определитель матрицы также равен 0 $\Rightarrow$ среди собственных значений будет 0 для $n>=2$.

Мы получаем последовательность, что для любого $n$ собственные значения будут состоять из $n-1$ нулей и одного значения $a_1^2 + a_2^2 + c_3^2+...+a^2_n$.
Длинная конструкция с квадратами похожа на квадрат длины вектора $x$. Покажем, что это верно, используя свойства собственного значения. Предположим, что мы нашли собственный вектор $x$ (вектор столбец), тогда выполнится определение $Ax=\lambda x$.


$$(x^Tx)x^T = x^T(xx^T)=x^T{||x||}^2={||x||}^2x^T=\lambda x^T$$

и $\lambda={||x||}^2$, следовательно квадрат длины вектора $x$ всегда будет собственным значением.

\textbf{Ответ: собственным значением будет квадрат длины вектора $x$: ${||x||}^2=a_1^2 + a_2^2 + c_3^2+...+a^2_n$ и $n-1$ штук нулей.}

\end{enumerate}
\end{document}