\documentclass[a4paper,12pt]{article}
\usepackage[utf8]{inputenc}
\usepackage[cm,empty]{fullpage}
\usepackage[T2A]{fontenc}
\usepackage[english, russian]{babel}
\usepackage{amssymb,amsmath,amsxtra,amsthm}
\usepackage{proof}
\usepackage[pdftex]{graphicx}
\usepackage{wrapfig}
\usepackage{braket}
\usepackage{xcolor}
\usepackage{enumitem}

\usepackage[left=2cm,right=2cm,
    top=1cm,bottom=1cm,bindingoffset=0cm]{geometry}

\renewcommand{\leq}{\leqslant}
\renewcommand{\geq}{\geqslant}


\newcommand{\iiff}{\Longleftrightarrow}
\renewcommand{\iff}{\Leftrightarrow}
\newcommand{\nothing}{\varnothing}

\newtheorem*{rem}{Замечание}

\newcommand{\NN}{\mathbb{N}}
\newcommand{\ZZ}{\mathbb{Z}}
\newcommand{\Q}{\mathbb{Q}}
\newcommand{\A}{\mathbb{A}}
\newcommand{\R}{\mathbb{R}}
\renewcommand{\C}{\mathbb{C}}

\renewcommand{\phi}{\varphi}
\newcommand{\eps}{\varepsilon}

\makeatletter
\newcommand*{\rom}[1]{\expandafter\@slowromancap\romannumeral #1@}
\makeatother

\newcounter{z}


\newcommand{\zs}{\refstepcounter{z}\vskip 10pt\par\noindent
\fbox{\textbf{12.\arabic{z}}} }

\newcommand{\z}{\refstepcounter{z}\vskip 20pt\noindent
\fbox{\textbf{\arabic{z}}} }

\renewcommand{\date}{{\bf 5 апреля 2021}} 

\newcommand{\dif}
{
------------------------------------------------------------------------------------------------------------------------------------------------------
}

\newcommand{\HSEhat}{
\vspace*{-0pt}
\noindent
\setcounter{z}{0}


{\bf \phantom{\date}  \large \hfill Линейная алгебра: \hfill \normalsize \date}

\vspace{5 pt}
{\bf \large \hfill  лекция 4\hfill }

\vspace{15 pt}
\centerline{ \large  Домашнее задание.}
\centerline{ \large  Кирилл Сетдеков}



\vspace*{10pt}
\setcounter{z}{0}

}

\begin{document}
\HSEhat





Квадратные матрицы $A, B\in M_{n}(\mathbb{R})$ называются сопряженными, если найдется невырожденная матрица $C\in M_{n}(\mathbb{R})$ такая, что $B = C^{-1}AC$. Сопряженные матрицы обладают похожими свойствами, в частности, имеют одинаковый определитель, след и ранг, а также одинаковые наборы собственных значений.

В пространстве $\mathbb R^n$ стандартным называется следующий базис
\[
e_1 = 
\begin{pmatrix}
{1}\\{0}\\{\vdots}\\{0}
\end{pmatrix},
e_2 = 
\begin{pmatrix}
{0}\\{1}\\{\vdots}\\{0}
\end{pmatrix},
\ldots,
e_n = 
\begin{pmatrix}
{0}\\{0}\\{\vdots}\\{1}
\end{pmatrix}
\]


\begin{enumerate}

\subsection*{Задачи:}




\item Какие из следующих матриц сопряжены? Если они сопряжены, то укажите, с помощью какой матрицы:
\begin{itemize}
\item[а)] $\begin{pmatrix}{1}&{0}\\{0}&{0}\end{pmatrix}$ и $\begin{pmatrix}{0}&{0}\\{0}&{1}\end{pmatrix};$

\textbf{Решение:}\\
Найдем определитель:
$$D=b^2-4ac =  (\frac{20}{i - 3})^2+4\frac{2 - 11i}i=\frac{400}{i^2+9-6i}+\frac{8-44i}{i}=\frac{400}{8-6i}+\frac{8-44i}{i}=$$ $$=\frac{400i+64-264-48i-352i}{8i+6}=\frac{-200}{8i+6}=-100 (\frac{1}{4i+3})=-100(\frac{3}{25}-\frac{4i}{25})=-12+16i=D$$

$$\sqrt{D}=\sqrt{4}\sqrt{-3+4i}=\sqrt{4}(\sqrt{\frac{\sqrt{25}-3}{2}}+i\sqrt{\frac{\sqrt{25}+3}{2}})=\sqrt{4}+4i=2+4i$$

Решение будет:
$$x_{1,2} = \frac{-b\pm (2+4i)}{2a}=\frac{\frac{20}{3-i}\pm (2+4i)}{\frac{2}{-i}}=-10\frac{i}{3-i}\pm i (1+2i)=-10\frac{i}{3-i}\pm(i-2) =$$
$$= -10i(\frac{3}{10}+\frac{i}{10})\pm(i-2)=-3i+1 \pm(i-2)$$
После того как возьмем два разных знака, то получим корни:
$x_1 = 3-4i; x_2=-1-2i$


\textbf{Ответ: $x_1 = 3-4i; x_2=-1-2i$}



\item[б)] $\begin{pmatrix}{3}&{1}\\{1}&{2}\end{pmatrix}$ и $\begin{pmatrix}{3}&{-1}\\{2}&{1}\end{pmatrix};$
\item[в)] $\begin{pmatrix}{2}&{1}\\{1}&{1}\end{pmatrix}$ и $\begin{pmatrix}{2}&{1}\\{0}&{1}\end{pmatrix}$.
\end{itemize}

\vspace{5pt}


\item В пространстве $\mathbb R^3$ заданы следующие векторы:
\[
v_1 =
\begin{pmatrix}
{1}\\{0}\\{2}
\end{pmatrix}
,\;
v_2 =
\begin{pmatrix}
{1}\\{1}\\{-1}
\end{pmatrix}
,\;
v_3 =
\begin{pmatrix}
{2}\\{0}\\{3}
\end{pmatrix}
,\;
u_1 =
\begin{pmatrix}
{1}\\{0}\\{1}
\end{pmatrix}
,\;
u_2 =
\begin{pmatrix}
{0}\\{1}\\{1}
\end{pmatrix}
,\;
u_3 =
\begin{pmatrix}
{0}\\{0}\\{2}
\end{pmatrix}.
\]
Найдите матрицу $A$ линейного оператора $\varphi\colon \mathbb R^3\to \mathbb R^3$, заданного по правилу $x\mapsto Ax$, такого, что $A v_i = u_i$ при $i = 1, 2, 3$. 

\vspace{5pt}



\item Пусть $\varphi\colon \mathbb R^3\to \mathbb R^2$ -- линейное отображение, заданное в стандартном базисе матрицей $A = \left(\begin{smallmatrix}{1}&{2}&{0}\\{-1}&{0}&{2}\end{smallmatrix}\right)$.  Пусть
\[
f_1 = 
\begin{pmatrix}
{1}\\{1}\\{1}
\end{pmatrix},\:
f_2 = 
\begin{pmatrix}
{1}\\{1}\\{2}
\end{pmatrix},\:
f_3 = 
\begin{pmatrix}
{1}\\{2}\\{3}
\end{pmatrix}\text{ -- \ векторы в } \mathbb R^3,\quad
g_1 =
\begin{pmatrix}
{1}\\{2}
\end{pmatrix},\:
g_2 =
\begin{pmatrix}
{1}\\{1}
\end{pmatrix}\text{ -- \ векторы в } \mathbb R^3.
\]
Найти матрицу отображения $\varphi$ в базисах $f_1,f_2,f_3$ и $g_1,g_2$.

\vspace{5pt}



\item В пространстве $\mathbb R^3$ заданы следующие векторы:
\[
v_1 = 
\begin{pmatrix}
{1}\\{1}\\{1}
\end{pmatrix},\,
v_2 = 
\begin{pmatrix}
{1}\\{2}\\{0}
\end{pmatrix},\,
v_3 = 
\begin{pmatrix}
{5}\\{8}\\{2}
\end{pmatrix},\,
v_4 = 
\begin{pmatrix}
{-1}\\{-2}\\{1}
\end{pmatrix},\,
v_5 = 
\begin{pmatrix}
{3}\\{5}\\{-1}
\end{pmatrix}.
\]
Существует ли линейное отображение $\varphi\colon \mathbb R^3\to \mathbb R^2$ такое, что $\varphi(v_i) = u_i$ при $i = 1, 2, 3, 4, 5$, где

\[
u_1 = 
\begin{pmatrix}
{1}\\{0}
\end{pmatrix},\,
u_2 = 
\begin{pmatrix}
{0}\\{1}
\end{pmatrix},\,
u_3 = 
\begin{pmatrix}
{2}\\{3}
\end{pmatrix},\,
u_4 = 
\begin{pmatrix}
{0}\\{0}
\end{pmatrix},\,
u_5 = 
\begin{pmatrix}
{1}\\{0}
\end{pmatrix}?
\]

\vspace{5pt}

\item Найдите в стандартном базисе матрицу линейного оператора $\varphi\colon \mathbb R^2\to \mathbb R^2$, который сжимает плоскость в 2 раза вдоль прямой $y = -2x$ и одновременно растягивает плоскость в 3 раза вдоль прямой $y = x$.

\vspace{5pt}


\item Найти собственные значения и собственные векторы линейного оператора, заданного в стандартном базисе матрицей:



$$
\begin{pmatrix}
{4}&{-5}&{2}\\
{5}&{-7}&{3}\\
{6}&{-9}&{4}\\
\end{pmatrix}
$$

Можно ли эти матрицу диагонализовать в каком-нибудь базисе?

\vspace{5pt}


\item Пусть $a_1, a_2, \dots, a_n \in \mathbb{R}$. Найдите собственные значения $n \times n$ матрицы $x^T x$, где $x$ -- матрица-строка $(a_1, a_2,\ldots,a_n)$.


\end{enumerate}
\end{document}