\documentclass[a4paper,12pt]{article}
\usepackage[utf8]{inputenc}
\usepackage[cm,empty]{fullpage}
\usepackage[T2A]{fontenc}
\usepackage[english, russian]{babel}
\usepackage{amssymb,amsmath,amsxtra,amsthm}
\usepackage{proof}
\usepackage[pdftex]{graphicx}
\usepackage{wrapfig}
\usepackage{braket}
\usepackage{xcolor}
\usepackage{enumitem}

\usepackage[left=2cm,right=2cm,
    top=1cm,bottom=1cm,bindingoffset=0cm]{geometry}

\renewcommand{\leq}{\leqslant}
\renewcommand{\geq}{\geqslant}


\newcommand{\iiff}{\Longleftrightarrow}
\renewcommand{\iff}{\Leftrightarrow}
\newcommand{\nothing}{\varnothing}

\newtheorem*{rem}{Замечание}

\newcommand{\NN}{\mathbb{N}}
\newcommand{\ZZ}{\mathbb{Z}}
\newcommand{\Q}{\mathbb{Q}}
\newcommand{\A}{\mathbb{A}}
\newcommand{\R}{\mathbb{R}}
\renewcommand{\C}{\mathbb{C}}

\renewcommand{\phi}{\varphi}
\newcommand{\eps}{\varepsilon}

\makeatletter
\newcommand*{\rom}[1]{\expandafter\@slowromancap\romannumeral #1@}
\makeatother

\newcounter{z}


\newcommand{\zs}{\refstepcounter{z}\vskip 10pt\par\noindent
\fbox{\textbf{12.\arabic{z}}} }

\newcommand{\z}{\refstepcounter{z}\vskip 20pt\noindent
\fbox{\textbf{\arabic{z}}} }

\renewcommand{\date}{{\bf 15 апреля 2021}} 

\newcommand{\dif}
{
------------------------------------------------------------------------------------------------------------------------------------------------------
}

\newcommand{\HSEhat}{
\vspace*{-0pt}
\noindent
\setcounter{z}{0}


{\bf \phantom{\date}  \large \hfill Линейная алгебра: \hfill \normalsize \date}

\vspace{5 pt}
{\bf \large \hfill  лекция 5\hfill }

\vspace{15 pt}
\centerline{ \large  Домашнее задание.}
\centerline{ \large  Кирилл Сетдеков}



\vspace*{10pt}
\setcounter{z}{0}

}

\begin{document}
\HSEhat


\begin{enumerate}

\subsection*{Задачи:}


\item
Найдите жорданову нормальную форму (ЖНФ) следующей матрицы:
$$
\begin{pmatrix}
0 & 0 & 1 & 0\\
0 & 0 & 0 & 1\\
3 & 4 & 0 & 0\\
-1 & -1 & 0 & 0
\end{pmatrix}.
$$
\vspace{5pt}

\textbf{Решение:}\\
Найдем собственные значения через $|A-\lambda E|=0$

$\begin{vmatrix}
-\lambda & 0 & 1 & 0\\
0 & -\lambda & 0 & 1\\
3 & 4 & -\lambda & 0\\
-1 & -1 & 0 & -\lambda
\end{vmatrix} =$ <разложим по 1 строке>$=-\lambda \begin{vmatrix}
 -\lambda & 0 & 1\\
 4 & -\lambda & 0\\
 -1 & 0 & -\lambda
\end{vmatrix} + \begin{vmatrix}
0 & -\lambda  & 1\\
3 & 4  & 0\\
-1 & -1  & -\lambda
\end{vmatrix} =$<разложим по 2 столбцу 1 определитель и по 1 строке второй определитель> = $\lambda^2\begin{vmatrix}
 -\lambda  & 1\\
 -1  & -\lambda
\end{vmatrix}+\lambda \begin{vmatrix}
 3 & 0\\
 -1  & -\lambda
\end{vmatrix}+ \begin{vmatrix}
 3 & 4\\
 -1  & -1
\end{vmatrix} = \lambda^4+\lambda^2-3\lambda^2+1=\lambda^4-2\lambda^2+1$

Характеристических многочлен $\lambda^4-2\lambda^2+1=(\lambda^2 -1)^2$. Корнем будет являться $\lambda^2 = 1 \Rightarrow$ $\lambda_1 = -1, \lambda_2 = 1$

Каждый корень кратности 2, поэтому мы не знаем точно форму клетки, в которую каждый входит. Проверим для каждого корня число клеток 1x1, это позволит нам понять, образуют ли они клетки типа $\begin{pmatrix}
\lambda & 0\\
0& \lambda
\end{pmatrix}$ если число=2 или вида $\begin{pmatrix}
\lambda & 1\\
0& \lambda
\end{pmatrix}$ если число=0.

Проверим для $\lambda_1 = -1$:

Нам нужно будет рассчитать 2 ранга:

Для того чтобы вставить $rk(A+E)^2$, посчитаем его.


$(A+E)^2=\begin{pmatrix}
1 & 0 & 1 & 0\\
0 & 1 & 0 & 1\\
3 & 4 & 1 & 0\\
-1 & -1 & 0 & 1
\end{pmatrix}\begin{pmatrix}
1 & 0 & 1 & 0\\
0 & 1 & 0 & 1\\
3 & 4 & 1 & 0\\
-1 & -1 & 0 & 1
\end{pmatrix}=\begin{pmatrix}
4 & 4 & 2 & 0\\
-1 & 0 & 0 & 2\\
6 & 8 & 4 & 4\\
-2 & -2 & -1 & 0
\end{pmatrix}$\\
Приведем эту матрицу к ступенчатому виду:\\
$
\left(\begin{array}{cccc}  
4&4&2&0\\
-1&0&0&2\\
6&8&4&4\\
-2&-2&-1&0\\
\end{array}\right) \rightarrow$ поменяем \rom{1} и \rom{2} местами, поделим \rom{2} и \rom{3} на 2 \\
$
\left(\begin{array}{cccc}  
-1&0&0&2\\
2&2&1&0\\
3&4&2&2\\
-2&-2&-1&0\\
\end{array}\right) \rightarrow$  \rom{2} + 2 \rom{1}, \rom{3} + 3 \rom{1}, \rom{4} - 2 \rom{1} \\
$
\left(\begin{array}{cccc}  
-1&0&0&2\\
0&2&1&4\\
0&4&2&8\\
0&-2&-1&-4\\
\end{array}\right) \rightarrow$  \rom{3} - 2 \rom{3}, \rom{4} + \rom{2} \\
$
\left(\begin{array}{cccc}  
-1&0&0&2\\
0&2&1&4\\
0&0&0&0\\
0&0&0&0\\
\end{array}\right) $   В этой матрице 2 главных переменных. Ее ранг = 2.

Найдем ранг матрицы $A+E = \begin{pmatrix}
1 & 0 & 1 & 0\\
0 & 1 & 0 & 1\\
3 & 4 & 1 & 0\\
-1 & -1 & 0 & 1
\end{pmatrix}$\rom{3} - 3 \rom{1}, \rom{4} + \rom{1} $\rightarrow$\\
$
\left(\begin{array}{cccc}  
1&0&1&0\\
0&1&0&1\\
0&4&-2&0\\
0&-1&1&1\\
\end{array}\right) \rightarrow$  \rom{3} - 4 \rom{2}, \rom{4} + \rom{2} \\
$
\left(\begin{array}{cccc}  
1&0&1&0\\
0&1&0&1\\
0&0&-2&-4\\
0&0&1&2\\
\end{array}\right) \rightarrow$  \rom{3} - 2 \rom{4} \\
$
\left(\begin{array}{cccc}  
1&0&1&0\\
0&1&0&1\\
0&0&0&0\\
0&0&1&2\\
\end{array}\right) $  В этой матрице 3 главных переменных. Ее ранг = 3.

$N(-1,1)=rk(A+E)^0+rk(A+E)^2-2rk(A+E) =  4+2-2*3=0$ Для этого корня число жардановых клеток размером 1 равно 0.



Проверим для $\lambda_2 = 1$:
Нам нужно будет рассчитать 2 ранга:

Для того чтобы вставить $rk(A-E)^2$, посчитаем его.

$(A-E)^2=\begin{pmatrix}
-1&0&1&0\\
0&-1&0&1\\
3&4&-1&0\\
-1&-1&0&-1\\
\end{pmatrix}\begin{pmatrix}
-1&0&1&0\\
0&-1&0&1\\
3&4&-1&0\\
-1&-1&0&-1\\
\end{pmatrix}=\begin{pmatrix}
4&4&-2&0\\
-1&0&0&-2\\
-6&-8&4&4\\
2&2&-1&0\\
\end{pmatrix}   \rom{1} \leftrightarrow \rom{2}$ \\
$
\left(\begin{array}{cccc}  
-1&0&0&-2\\
4&4&-2&0\\
-6&-8&4&4\\
2&2&-1&0\\
\end{array}\right) \rightarrow$  \rom{2} + 4 \rom{1}, \rom{3} + 6 \rom{1}, \rom{4} + 2 \rom{1}\\
$
\left(\begin{array}{cccc}  
-1&0&0&-2\\
0&4&-2&-8\\
0&-8&4&16\\
0&2&-1&-4\\
\end{array}\right) \rightarrow$  \rom{3} + 2 \rom{2}, \rom{4} - 0.5 \rom{2}\\
$
\left(\begin{array}{cccc}  
-1&0&0&-2\\
0&4&-2&-8\\
0&0&0&0\\
0&0&0&0\\
\end{array}\right) \Rightarrow$ 2 главных переменных, Ранг этой матрицы 2.

Найдем ранг матрицы $A-E = \begin{pmatrix}
-1&0&1&0\\
0&-1&0&1\\
3&4&-1&0\\
-1&-1&0&-1\\
\end{pmatrix}$ \rom{3} + 3 \rom{1}, \rom{4} - \rom{1}\\
$
\left(\begin{array}{cccc}  
-1&0&1&0\\
0&-1&0&1\\
0&4&2&0\\
0&-1&-1&-1\\
\end{array}\right) \rightarrow$  \rom{3} + 4 \rom{2}, \rom{4} -  \rom{2}\\
$
\left(\begin{array}{cccc}  
-1&0&1&0\\
0&-1&0&1\\
0&0&2&4\\
0&0&-1&-2\\
\end{array}\right) \rightarrow$  \rom{4} + 0.5 \rom{3}\\
$
\left(\begin{array}{cccc}  
-1&0&1&0\\
0&-1&0&1\\
0&0&2&4\\
0&0&0&0\\
\end{array}\right) \Rightarrow$  3 главных переменных, ранг этой матрицы 3


$N(1,1)=rk(A-E)^0+rk(A-E)^2-2rk(A-E) =4+2-2*3=0$  Для этого корня число жардановых клеток размером 1 равно 0.

Следовательно для каждого корня жарданова клетка будет иметь вид $\begin{pmatrix}
\lambda & 1\\
0& \lambda
\end{pmatrix}$. Расположим их по диагонали матрицы, подставив собственные значения и запишем ответ:

\textbf{Ответ: ЖНФ исходной матрицы: $\begin{pmatrix}
-1 & 1 & 0 & 0\\
0 & -1 & 0 & 0\\
0 & 0 & 1 & 1\\
0 & 0 & 0 & 1
\end{pmatrix}$}

\item Найдите длины сторон и внутренние углы треугольника $ABC$ в пространстве $\mathbb R^5$ со стандартным скалярным произведением, если координаты вершин треугольника таковы:
\[
A = 
\begin{pmatrix}
{2}\\{4}\\{2}\\{4}\\{2}
\end{pmatrix}
,\quad
B = 
\begin{pmatrix}
{6}\\{4}\\{4}\\{4}\\{6}
\end{pmatrix}
,\quad
C = 
\begin{pmatrix}
{5}\\{7}\\{5}\\{7}\\{2}
\end{pmatrix}
\]
\vspace{5pt}

\textbf{Решение:}\\
Сначала угол $A$. Найдем вектор сторон, которые его образуют, вычтя координаты точек. $AB = \begin{pmatrix}
4\\0\\2\\0\\4
\end{pmatrix}$, $AC = \begin{pmatrix}
3\\3\\3\\3\\0
\end{pmatrix}$. Длина вектора равна $|v| = \sqrt{(v,v)}$. Длина $|AB| = \sqrt{36}=6$, длина $|AC| = \sqrt{36}=6$. Угол найдем через скалярное произведение, деленное на произведение длин векторов. $\cos{A}= \frac{(AB,AC)}{|AB| |AC|}=\frac{18}{36}=0.5$. $A=\arccos{0.5} = \pi/3=60^{\circ}$

Аналогично найдем угол $B$.
$BC = \begin{pmatrix}
-1\\3\\1\\3\\-4
\end{pmatrix}$, $BA=-AB = \begin{pmatrix}
-4\\0\\-2\\0\\-4
\end{pmatrix}$. Длина $|BC| = \sqrt{36}=6$, длина $|BA|=6$. Угол найдем через $\cos{B}= \frac{(BA,BC)}{|BA| |BC|}=\frac{18}{36}=0.5$. $B=60^{\circ}$

Аналогично найдем угол $C$.
$CA=-AC = \begin{pmatrix}
-3\\-3\\-3\\-3\\0
\end{pmatrix}$, $CB=-BC = \begin{pmatrix}
1\\-3\\-1\\-3\\4
\end{pmatrix}$. Длина $|CA|  =6$, длина $|CB|=6$. Угол найдем через $\cos{C}= \frac{(CA,CB)}{|CA| |CB|}=\frac{18}{36}=0.5$. $C=60^{\circ}$

\textbf{Ответ: Это правильный  треугольник, у него равны стороны и углы: Угол $A=B=C = \pi/3=60^{\circ}$. Стороны равны и имеют длину 6: $|AB|=|BC|=|AC|=6$}

\item Пусть в $\mathbb R^5$ задано стандартное скалярное произведение. И пусть 
\[
x =
\begin{pmatrix}
{1}\\{-5}\\{6}\\{4}\\{5}
\end{pmatrix}
,\quad
a_1 = 
\begin{pmatrix}
{7}\\{5}\\{0}\\{1}\\{0}
\end{pmatrix}
,\quad
a_2 = 
\begin{pmatrix}
{-4}\\{1}\\{5}\\{6}\\{0}
\end{pmatrix}
,\quad
a_3 = 
\begin{pmatrix}
{-3}\\{-2}\\{4}\\{-4}\\{0}
\end{pmatrix}, \quad
a_4 = 
\begin{pmatrix}
{3}\\{1}\\{0}\\{0}\\{-3}
\end{pmatrix}.
\]

Найдите расстояние от вектора $x$ до подпространства $U = \langle a_1,a_2,a_3, a_4 \rangle$, а также проекцию вектора $x$ на $U$.
\vspace{5pt}

\textbf{Решение:}\\
сначала найдем вектор нормали $v$, который ортогональный каждому и из векторов в $U$. Для этого нужно решить систему уравнений относительно координат этого вектора, такую, что скалярное произведение $(v,a_i)=0$ для любого $i \in [1,2,3,4]$.

$$\left\{
                \begin{array}{ll}
                  7v_1+5v_2+1v_4=0\\
                  -4v_1+v_2+5v_3+6v_4=0\\
                  -3v_1-2v_2+4v_3-4v_4 = 0\\
                  3v_1+v_1-3v_5=0
                \end{array}
              \right.$$
              
Запишем коэффициенты этой системы в матрицу и найдем решение методом Гаусса.

$\begin{pmatrix}
7&5&0&1&0\\
-4&1&5&6&0\\
-3&-2&4&-4&0\\
3&1&0&0&-3\\
\end{pmatrix}$\rom{3} -  \rom{2}, \rom{3} переместить на 1 строку $\rightarrow$\\
$\begin{pmatrix}
1&-3&-1&-10&0\\
7&5&0&1&0\\
-4&1&5&6&0\\
3&1&0&0&-3\\
\end{pmatrix}$\rom{2} - 7 \rom{1}, \rom{3} +4  \rom{1}, \rom{4} - 3 \rom{1}$\rightarrow$\\
$\begin{pmatrix}
1&-3&-1&-10&0\\
0&26&7&71&0\\
0&-11&1&-34&0\\
0&10&3&30&-3\\
\end{pmatrix}$-\rom{4} -  \rom{3}$\rightarrow$\\
$\begin{pmatrix}
1&-3&-1&-10&0\\
0&26&7&71&0\\
0&-11&1&-34&0\\
0&1&-4&4&3\\
\end{pmatrix}$\rom{1} +3  \rom{4}, \rom{2} -26   \rom{4},\rom{3} +11  \rom{4},$\rightarrow$\\
$\begin{pmatrix}
1&0&-13&2&9\\
0&0&111&-33&-78\\
0&0&-43&10&33\\
0&1&-4&4&3\\
\end{pmatrix}$\rom{2} поменяем   \rom{4}$\rightarrow$\\
$\begin{pmatrix}
1&0&-13&2&9\\
0&1&-4&4&3\\
0&0&-43&10&33\\
0&0&111&-33&-78\\
\end{pmatrix}$\rom{4} +2    \rom{3}$\rightarrow$\\
$\begin{pmatrix}
1&0&-13&2&9\\
0&1&-4&4&3\\
0&0&-43&10&33\\
0&0&25&-13&-12\\
\end{pmatrix}$\rom{3}*25,    \rom{4}*43$\rightarrow$\\
$\begin{pmatrix}
1&0&-13&2&9\\
0&1&-4&4&3\\
0&0&-1075&250&825\\
0&0&1075&-559&-516\\
\end{pmatrix}$\rom{4}+    \rom{3}$\rightarrow$\\
$\begin{pmatrix}
1&0&-13&2&9\\
0&1&-4&4&3\\
0&0&-1075&250&825\\
0&0&1075&-559&-516\\
\end{pmatrix}$\rom{4}+    \rom{3}$\rightarrow$\\
$\begin{pmatrix}
1&0&-13&2&9\\
0&1&-4&4&3\\
0&0&-1075&250&825\\
0&0&0&-309&309\\
\end{pmatrix}$\rom{4}:-309$\rightarrow$\\
$\begin{pmatrix}
1&0&-13&2&9\\
0&1&-4&4&3\\
0&0&-1075&250&825\\
0&0&0&1&-1\\
\end{pmatrix}$\rom{1}-2*\rom{4},\rom{2}-4*\rom{4},\rom{3}-250*\rom{4}$\rightarrow$\\
$\begin{pmatrix}
1&0&-13&0&11\\
0&1&-4&0&7\\
0&0&-1075&0&1075\\
0&0&0&1&-1\\
\end{pmatrix}$\rom{3}:-1075$\rightarrow$\\
$\begin{pmatrix}
1&0&-13&0&11\\
0&1&-4&0&7\\
0&0&1&0&-1\\
0&0&0&1&-1\\
\end{pmatrix}$\rom{1}+13 \rom{3}, \rom{2}+4 \rom{3}$\rightarrow$\\
$\begin{pmatrix}
1&0&0&0&-2\\
0&1&0&0&3\\
0&0&1&0&-1\\
0&0&0&1&-1\\
\end{pmatrix}$. Мы можем выписать ответ. $v = \begin{pmatrix}
2t\\-3t\\t\\t\\t
\end{pmatrix}=t\begin{pmatrix}
2\\-3\\1\\1\\1
\end{pmatrix}$

Расстояние от вектора до подпространства будет длиной проекции x на v. Найдем проекцию x на v:
$pr_{v}x = \frac{(x,v)}{(v,v)}v = \frac{32}{16}\begin{pmatrix}
2\\-3\\1\\1\\1
\end{pmatrix}=\begin{pmatrix}
4\\-6\\2\\2\\2
\end{pmatrix}$\\
$$\rho(x,U) = |pr_{v}x|=\sqrt{64}=8$$
$$pr_{U}x= x - pr_{v}x =\begin{pmatrix}
{1}\\{-5}\\{6}\\{4}\\{5}
\end{pmatrix}-\begin{pmatrix}
4\\-6\\2\\2\\2
\end{pmatrix}=\begin{pmatrix}
-3\\1\\4\\2\\3
\end{pmatrix} $$

\textbf{Ответ: расстояние от x до U: $\rho(x,U) =8$ и проекция x на U: $pr_{U}x= \begin{pmatrix}
-3\\1\\4\\2\\3
\end{pmatrix} $ }

\item Применяя процесс ортогонализации Грама--Шмидта, постройте ортогональный базис линейной оболочки  $\langle a_1,a_2,a_3\rangle$ евклидова пространства $\mathbb R^4$ со стандартным скалярным произведением, где
\[
a_1=
\begin{pmatrix}
{1}\\{2}\\{2}\\{-1}
\end{pmatrix}
,\quad
a_2=
\begin{pmatrix}
{1}\\{1}\\{-5}\\{3}
\end{pmatrix}
,\quad
a_3=
\begin{pmatrix}
{3}\\{2}\\{8}\\{-7}
\end{pmatrix}
\]
\vspace{5pt}

\textbf{Решение:}\\
Шаг 1) возьмем  $u_1 = a_1=\begin{pmatrix}
{1}\\{2}\\{2}\\{-1}
\end{pmatrix}$

Шаг 2) $u_2 = a_2 - pr_{u_1}a_2= \begin{pmatrix}
{1}\\{1}\\{-5}\\{3}
\end{pmatrix} - \frac{-10}{10} \begin{pmatrix}
{1}\\{2}\\{2}\\{-1}
\end{pmatrix} = \begin{pmatrix}
{2}\\{3}\\{-3}\\{2}
\end{pmatrix}$

Шаг 3) $u_3 = a_3  - pr_{u_1}a_3  - pr_{u_2}a_3 = a_3 - \frac{(a_3, u_1)}{(u_1, u_1)} u_1- \frac{(a_3, u_2)}{(u_2, u_2)} u_2 = \begin{pmatrix}
{3}\\{2}\\{8}\\{-7}
\end{pmatrix}-\frac{30}{10} \begin{pmatrix}
{1}\\{2}\\{2}\\{-1}
\end{pmatrix}- \frac{-26}{26} \begin{pmatrix}
{2}\\{3}\\{-3}\\{2}
\end{pmatrix}=\begin{pmatrix}
{2}\\{-1}\\{-1}\\{-2}
\end{pmatrix}$

\textbf{Ответ: ортогональный базис линейной оболочки  $\langle a_1,a_2,a_3\rangle$ равен:\\ $u_1 = \begin{pmatrix}
{1}\\{2}\\{2}\\{-1}
\end{pmatrix}$, $u_2 =\begin{pmatrix}
{2}\\{3}\\{-3}\\{2}
\end{pmatrix}$, $u_3 = \begin{pmatrix}
{2}\\{-1}\\{-1}\\{-2}
\end{pmatrix}$}


\item Пусть в пространстве $\mathbb R^3$ задано стандартное скалярное произведение, рассмотрим три точки:
\[
P_1 = 
\begin{pmatrix}
{1}\\{4}\\{0}
\end{pmatrix},\;\;
P_2 = 
\begin{pmatrix}
{-1}\\{4}\\{4}
\end{pmatrix}\text{ и }
P_3 = 
\begin{pmatrix}
{6}\\{8}\\{0}
\end{pmatrix}
\]

а) Найдите такие векторы $v, p\in \mathbb R^3$, что гиперповерхность $L = \{y\in \mathbb R^3 \mid (v, y) = (v, p)\}$ проходит через точки $P_1, P_2, P_3$.

\textbf{Решение:}\\
Запишем векторы, которые лежат между точками попарно.

$p_1= P_1P_2 = 
\begin{pmatrix}
{-1}\\{4}\\{4}
\end{pmatrix}-\begin{pmatrix}
{1}\\{4}\\{0}
\end{pmatrix}=\begin{pmatrix}
{-2}\\{0}\\{4}
\end{pmatrix}$

$p_2= P_1P_3 = 
\begin{pmatrix}
{6}\\{8}\\{0}
\end{pmatrix}-\begin{pmatrix}
{1}\\{4}\\{0}
\end{pmatrix}=\begin{pmatrix}
{5}\\{4}\\{0}
\end{pmatrix}$

Найдем вектор $v$, который будет нормальный $p_1$ и $p_2$
Поступим аналогично заданию 3 и решим методом Гаусса

$\begin{pmatrix}
-2&0&4\\
5&4&0\\
\end{pmatrix}$\rom{1}:-2 $\rightarrow$
$\begin{pmatrix}
1&0&-2\\
5&4&0\\
\end{pmatrix}$\rom{2}-5 \rom{2} $\rightarrow$$\begin{pmatrix}
1&0&-2\\
0&4&10\\
\end{pmatrix}$ из этих уравнений выразим координаты коллинеарных векторов, нормальных к поверхности.

$$v =\begin{pmatrix}
4t\\-5t\\2t
\end{pmatrix}=t\begin{pmatrix}
4\\-5\\2
\end{pmatrix}$$

Подставим, например $P_1 \rightarrow (v,y)$, получим $(v,y)=-16$. Мы хотим найти такой вектор сдвига, что $(v,p)=-16$. $(v,v) = 45 \Rightarrow p = -\frac{16}{45}v=\begin{pmatrix}
-\frac{64}{45}\\\frac{80}{45}\\-\frac{32}{45}
\end{pmatrix}$

\textbf{Ответ: $p =\begin{pmatrix}
-\frac{64}{45}\\\frac{80}{45}\\-\frac{32}{45}
\end{pmatrix}$, $v =\begin{pmatrix}
4\\-5\\2
\end{pmatrix}$ }

б) Выясните, на каком расстоянии от гиперповерхности $L$ лежат векторы
\[
w_1 =
\begin{pmatrix}
{1}\\{3}\\{5}
\end{pmatrix}
\text{ и \ }
w_2 = 
\begin{pmatrix}
{3}\\{8}\\{3}
\end{pmatrix}.
\]
По одну ли сторону от гиперповерхности $L$ они лежат?

\vspace{5pt}
\textbf{Решение:}\\
Посчитаем скалярные произведения $$(w_1,v)=-1 > -16 = (p,v)$$
$$(w_2,v)=-22 < -16 = (p,v)$$
Знак результаты находятся по разные стороны от -16 $\Rightarrow$ - эти векторы лежат по разные стороны от гиперплоскости

\textbf{Ответ: вектора $w_1, w_2$ лежат по разные стороны гиперплоскости из условия}

\item Методом наименьших квадратов найдите приближенное решение следующей системы линейных уравнений:
\[
\begin{cases}
2x - z = 1\\
 y + z = -1\\
x - y + z = 0\\
x - z = -1
\end{cases}
\]
\vspace{5pt}

\textbf{Решение:}\\
Запишем эту систему уравнений в матричной форме $AX = b$, где $X=\begin{pmatrix}
x\\y\\z
\end{pmatrix}$, $A=\begin{pmatrix}
2&0&-1\\
0&1&1\\
1&-1&1\\
1&0&-1\\
\end{pmatrix}$, $b=\begin{pmatrix}
1\\-1\\0\\-1
\end{pmatrix}$

Приближенное решение будет равно 
\[ 
\widehat{X}= \left(A^T A\right)^{-1}A^T b
\]


$$A^T A= \begin{pmatrix}
6&-1&-2\\
-1&2&0\\
-2&0&4
\end{pmatrix}$$

\[ 
\left(A^T A\right)^{-1} = \begin{pmatrix}
2/9& 1/9& 1/9\\
1/9& 5/9& 1/18\\
1/9& 1/18& 11/36
\end{pmatrix}
\]

\[ 
\left(A^T A\right)^{-1}A^T = \frac{1}{36}\begin{pmatrix}
12&8&8&4\\
6&22&-14&2\\
-3&13&13&-7
\end{pmatrix}
\]

\[ 
\widehat{X}= \left(A^T A\right)^{-1}A^T b=\frac{1}{36}\begin{pmatrix}
12&8&8&4\\
6&22&-14&2\\
-3&13&13&-7
\end{pmatrix} \begin{pmatrix}
1\\-1\\0\\-1
\end{pmatrix} = \begin{pmatrix}
0\\
-\frac{1}{2}\\
-\frac{1}{4}
\end{pmatrix}
\]

\textbf{Ответ: Приближенное решение системы уравнений $\widehat{x}=0, \widehat{y}=-\frac{1}{2}, \widehat{z}=-\frac{1}{4}$}

\end{enumerate}
\end{document}