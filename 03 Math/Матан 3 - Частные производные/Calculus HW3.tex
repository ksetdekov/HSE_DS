\documentclass[a4paper,12pt]{article}
\usepackage[utf8]{inputenc}
\usepackage[cm,empty]{fullpage}
\usepackage[T2A]{fontenc}
\usepackage[english, russian]{babel}
\usepackage{amssymb,amsmath,amsxtra,amsthm}
\usepackage{proof}
\usepackage[pdftex]{graphicx}
\usepackage{wrapfig}
\usepackage{braket}
\usepackage{xcolor}

\usepackage[left=2cm,right=2cm,
    top=1cm,bottom=1cm,bindingoffset=0cm]{geometry}

\renewcommand{\leq}{\leqslant}
\renewcommand{\geq}{\geqslant}


\newcommand{\iiff}{\Longleftrightarrow}
\renewcommand{\iff}{\Leftrightarrow}
\newcommand{\nothing}{\varnothing}

\newtheorem*{rem}{Замечание}

\newcommand{\NN}{\mathbb{N}}
\newcommand{\ZZ}{\mathbb{Z}}
\newcommand{\Q}{\mathbb{Q}}
\newcommand{\A}{\mathbb{A}}
\newcommand{\R}{\mathbb{R}}
\renewcommand{\C}{\mathbb{C}}

\renewcommand{\phi}{\varphi}
\newcommand{\eps}{\varepsilon}

\newcounter{z}


\newcommand{\zs}{\refstepcounter{z}\vskip 10pt\par\noindent
\fbox{\textbf{12.\arabic{z}}} }

\newcommand{\z}{\refstepcounter{z}\vskip 20pt\noindent
\fbox{\textbf{\arabic{z}}} }

\renewcommand{\date}{{\bf 6 марта 2021}} 

\newcommand{\dif}
{
------------------------------------------------------------------------------------------------------------------------------------------------------
}

\newcommand{\HSEhat}{
\vspace*{-0pt}
\noindent
\setcounter{z}{0}


{\bf \phantom{\date}  \large \hfill Математический анализ: \hfill \normalsize \date}

\vspace{5 pt}
{\bf \large \hfill  лекция 3\hfill }

\vspace{15 pt}
\centerline{ \large  Домашнее задание.}
\centerline{ \large  Кирилл Сетдеков}



\vspace*{10pt}
\setcounter{z}{0}

}

\begin{document}
\HSEhat


\subsection*{Задачи}

\begin{enumerate}

\item Найдите $\frac{\partial f}{\partial x} (a,1)$

$$f(x,y) =x + (y-1)\arcsin{\sqrt{\frac{x}{y}}}$$

\textbf{Решение:}\\
Дифференцируем как сумму:

$$\frac{\partial }{\partial x} (x + (y-1)\arcsin{\sqrt{\frac{x}{y}}}) = \frac{\partial }{\partial x} (x) +(y-1)(\frac{\partial }{\partial x} (\arcsin{\sqrt{\frac{x}{y}}})) = 1 + (y-1)(\frac{\partial }{\partial x} (\arcsin{\sqrt{\frac{x}{y}}})) = $$

Применяем цепное правило, где :
$\frac{\partial }{\partial x} (\arcsin{\sqrt{\frac{x}{y}}}) = \frac{\partial \arcsin{u}}{\partial u} \frac{\partial u}{\partial x}$ где $u = \sqrt{\frac{x}{y}} ;\frac{\partial \arcsin{u}}{\partial u} = \frac{1}{\sqrt{1-u^2}}$

$$=1 + (y-1)\frac{\frac{\partial }{\partial x}({\sqrt{\frac{x}{y}}}))}{\sqrt{1-\frac{x}{y}}} =$$

Применяем цепное правило для корня из частного:
$\frac{\partial }{\partial x}({\sqrt{\frac{x}{y}}})) = \frac{\partial \sqrt{u}}{\partial u} \frac{\partial u}{\partial x} $ где $u = \frac{y}{y}; \frac{\partial}{\partial u} (\sqrt{u}) = \frac{1}{2\sqrt{u}}$

и после этого - выносим константу и считаем производную

$$=1 +\frac{\frac{\partial }{\partial x}({{\frac{x}{y}}})}{2\sqrt{\frac{x}{y}}} \frac{y-1}{\sqrt{1-\frac{x}{y}}} =1 +\frac{1}{2y\sqrt{\frac{x}{y}}} \frac{y-1}{\sqrt{1-\frac{x}{y}}} = 1 + \frac{y-1}{2y\sqrt{\frac{x(y-x)}{y^2}}}=1 + \frac{y-1}{2\sqrt{x(y-x)}}$$

Подставим $x = a, y = 1$
$$1 + \frac{1-1}{2\sqrt{a(1-a)}} =1$$


\textbf{Ответ: $1$}


\item Найти градиент и матрицу Гессе
\begin{enumerate}
\item 
$
u = ln(x+y^2)
$

\textbf{Решение:}\\
Градиент по определению - ${grad}_au = (\frac{\partial u}{\partial x}(a), \frac{\partial u}{\partial y}(a))$

Матрица Гессе - квадратная матрица из производных второго порядка по x и y

$$H_a(u) = \begin{pmatrix}
\frac{\partial^2 u}{\partial x^2}(a) & \frac{\partial^2 u}{\partial x \partial y}(a)  \\
\frac{\partial^2 u}{\partial y \partial x}(a) & \frac{\partial^2 u}{\partial y^2}(a) 
\end{pmatrix}$$

Всего надо посчитать 6 производных, начнем с первой производной по х $$\frac{\partial ln(x+y^2)}{\partial x}$$

Используем цепное правило:
$\frac{\partial ln(x+y^2)}{\partial x} = \frac{\partial \ln{u}}{\partial u}\frac{\partial u}{\partial x}$ где $u = x + y^2, \frac{\partial}{\partial u} \ln{u} = \frac{1}{u}$

$$
= \frac{\frac{\partial}{\partial x}(x+y^2)}{x+y^2} = \frac{1}{x+y^2}
$$

Используя ту же замену, считаем: $$\frac{\partial ln(x+y^2)}{\partial y}=\frac{\frac{\partial}{\partial y}(x+y^2)}{x+y^2} = \frac{2y}{x+y^2}$$

\textbf{Ответ: ${grad}_au = (\frac{1}{x+y^2}(a), \frac{2y}{x+y^2}(a))$}

\item 
$
\int^{2\pi}_0 x \sin x dx
$

\textbf{Решение:}\\
ds

\textbf{Ответ: $-2\pi$}


\end{enumerate}

\item Найти предел $\lim_{x \to 0 }\frac{\int_{x}^{1} \frac{e^t}{t} dt}{\ln x}$

\textbf{Решение:}\\
Моё понимание, что интеграл в числителе будет равен разности двух интегральных показательных функций

$$
\lim_{x \to 0 }\frac{Ei(1) - Ei(x)}{\ln x}=
$$

Возьмем производную от верхней и нижней части 
$$= \lim_{x \to 0 }\frac{-e^x / x}{1/x} =\lim_{x \to 0 }-e^x = -1 $$

\textbf{Ответ: $-1$}

\end{enumerate}
\end{document}