\documentclass[a4paper,12pt]{article}
\usepackage[utf8]{inputenc}
\usepackage[cm,empty]{fullpage}
\usepackage[T2A]{fontenc}
\usepackage[english, russian]{babel}
\usepackage{amssymb,amsmath,amsxtra,amsthm}
\usepackage{proof}
\usepackage[pdftex]{graphicx}
\usepackage{wrapfig}

\usepackage[left=2cm,right=2cm,
    top=1cm,bottom=1cm,bindingoffset=0cm]{geometry}

\renewcommand{\leq}{\leqslant}
\renewcommand{\geq}{\geqslant}


\newcommand{\iiff}{\Longleftrightarrow}
\renewcommand{\iff}{\Leftrightarrow}
\newcommand{\nothing}{\varnothing}


\newcommand{\NN}{\mathbb{N}}
\newcommand{\ZZ}{\mathbb{Z}}
\newcommand{\Q}{\mathbb{Q}}
\newcommand{\A}{\mathbb{A}}
\newcommand{\R}{\mathbb{R}}
\renewcommand{\C}{\mathbb{C}}

\renewcommand{\phi}{\varphi}
\newcommand{\eps}{\varepsilon}

\newcounter{z}


\newcommand{\zs}{\refstepcounter{z}\vskip 10pt\par\noindent
\fbox{\textbf{12.\arabic{z}}} }

\newcommand{\z}{\refstepcounter{z}\vskip 20pt\noindent
\fbox{\textbf{\arabic{z}}} }

\renewcommand{\date}{{\bf 13 января 2021}} %Дата занятия

\newcommand{\dif}
{
------------------------------------------------------------------------------------------------------------------------------------------------------
}

\newcommand{\HSEhat}{
\vspace*{-0pt}
\noindent
\setcounter{z}{0}
\vspace*{-10pt}
\begin{wrapfigure}[2]{l}{5pt}
\vspace*{-25pt}
\hspace*{-20pt}
\includegraphics[scale=0.18]{img/HSE_LOGO.png}
\end{wrapfigure}
\vspace{-20pt}


{\bf \phantom{\date}  \large \hfill Дискретная математика: \hfill \normalsize \date}

\vspace{5 pt}
{\bf \large \hfill  множества и логика.\hfill }

\vspace{15 pt}
\centerline{ \large  Домашнее задание.}



\vspace*{10pt}
\setcounter{z}{0}

}

\begin{document}

\HSEhat

\z  Какие из следующих равенств выполнены для любых множеств $A$, $B$ и $C$?\\

{\bf а)} $A\setminus (A\cap B)= A \cap(A \setminus B);$\\

{\bf б)} $(A\cup B)\triangle (A\cap B)= A \triangle B;$\\

{\bf в)} $((A \backslash B) \cup(A \backslash C)) \cap(A \backslash(B \cap C))=A \backslash(B \cup C).$\\

Если равенство верно, то докажите его. Если не выполнено, то приведите контрпример.

\z Верно ли, что для любых множеств $A$ и $B$ выполняется включение

$$(A\cup B)\setminus B \subseteq A?$$

\z Докажите, что $\neg (a\vee (b \oplus 1))\wedge (a \rightarrow 1)=\neg a \wedge b.$

\z Для каких из ниже приведенных чисел ложно высказывание: <<Число четно $\wedge$ (В числе $7$ цифр $ \rightarrow \neg$(Третий разряд числа четный))>>?
\\

{\bf а)} $0$ \qquad  {\bf б)} $1234567,$ \qquad {\bf в)} $2222222,$  \qquad {\bf г)} $123457.$ 

\z Пусть $A=\{7,5,1,4,2,6,3\}, B= \{ x\ |\ x=2k,\ k\in \ZZ \}, C=\{0,1,2,3,4,5,6,7,8,9\}$. Для каких $x\in C$ предикат <<$(x \in A)\rightarrow \neg(x \in B)$>> обращается в истину?

\z Докажите, что сумма первых $n$ четных натуральных чисел равняется

$$2+4+6+8+\ldots +2n =n(n+1).$$ 

\end{document}