\documentclass[a4paper,12pt]{article}
\usepackage[utf8]{inputenc}
\usepackage[cm,empty]{fullpage}
\usepackage[T2A]{fontenc}
\usepackage[english, russian]{babel}
\usepackage{amssymb,amsmath,amsxtra,amsthm}
\usepackage{proof}
\usepackage[pdftex]{graphicx}
\usepackage{wrapfig}
\usepackage{braket}
\usepackage{xcolor}

\usepackage[left=2cm,right=2cm,
    top=1cm,bottom=1cm,bindingoffset=0cm]{geometry}

\renewcommand{\leq}{\leqslant}
\renewcommand{\geq}{\geqslant}


\newcommand{\iiff}{\Longleftrightarrow}
\renewcommand{\iff}{\Leftrightarrow}
\newcommand{\nothing}{\varnothing}

\newtheorem*{rem}{Замечание}

\newcommand{\NN}{\mathbb{N}}
\newcommand{\ZZ}{\mathbb{Z}}
\newcommand{\Q}{\mathbb{Q}}
\newcommand{\A}{\mathbb{A}}
\newcommand{\R}{\mathbb{R}}
\renewcommand{\C}{\mathbb{C}}

\renewcommand{\phi}{\varphi}
\newcommand{\eps}{\varepsilon}

\newcounter{z}


\newcommand{\zs}{\refstepcounter{z}\vskip 10pt\par\noindent
\fbox{\textbf{12.\arabic{z}}} }

\newcommand{\z}{\refstepcounter{z}\vskip 20pt\noindent
\fbox{\textbf{\arabic{z}}} }

\renewcommand{\date}{{\bf 27 февраля 2021}} %Дата занятия

\newcommand{\dif}
{
------------------------------------------------------------------------------------------------------------------------------------------------------
}

\newcommand{\HSEhat}{
\vspace*{-0pt}
\noindent
\setcounter{z}{0}


{\bf \phantom{\date}  \large \hfill Математический анализ: \hfill \normalsize \date}

\vspace{5 pt}
{\bf \large \hfill  лекция 2\hfill }

\vspace{15 pt}
\centerline{ \large  Домашнее задание.}
\centerline{ \large  Кирилл Сетдеков}



\vspace*{10pt}
\setcounter{z}{0}

}

\begin{document}
\HSEhat


\subsection*{Задачи}

\begin{enumerate}

\item Посчитайте неопределенные интегралы
\begin{enumerate}
\item 
$
\int sin^3 (x) dx
$

\textbf{Решение:}\\

$
\int {sin^3 (x) }dx = -\frac{1}{3} sin^2(x) cos(x) + \frac{2}{3} \int sin(x) dx = - \frac{2 cos (x)}{3} - \frac{1}{3} sin^2(x) cos(x) + C
$

\textbf{Ответ: $- \frac{2 \cos (x)}{3} - \frac{1}{3} \sin^2(x) \cos(x) + C$}


\item
$
\int{(\frac{1-x}{x})^2 dx}
$

\textbf{Решение:}\\
Проведем замену $u = 1-x$ и $du=-dx$

$
\int{(\frac{1-x}{x})^2 dx}=-\int{\frac{u^2}{(u-1)^2}du} = -\int{(\frac{2}{u-1}+\frac{1}{(u-1)^2}+1)du} = -2\int{\frac{1}{u-1}du} - \int{\frac{1}{(u-1)^2}du} -u +C
$

Проводим еще одну замену $s= u-1$ и $ds=du$

$ -2\int{\frac{1}{s}ds} - \int{\frac{1}{(s)^2}ds} -u +C
= -2\log s + \frac{1}{s} - u +C
$

Вернемся обратно к u и поток к x
$$=-2\log(u-1) + \frac{1}{u-1} -u + C = -2\log(-x) - \frac{1}{x} +x-1+ C=-2\log(-x) - \frac{1}{x} +x+ C$$

\textbf{Ответ: $-2\log(-x) - \frac{1}{x} +x+ C$}

\item 
$
\int\frac{\log^2 x}{x} dx
$

\textbf{Решение:}\\
Проведем замену $u = \log(x); du = \frac{1}{x}dx$


$$\int\frac{\log^2 x}{x} dx
= \int u^2 du = \frac{1}{3} u^3 + C = \frac{1}{3} \log^3 x + C $$

\textbf{Ответ: $\frac{1}{3} \log^3 x + C$}

\item 
$
\int\frac{1}{1+\cos x} dx
$

\textbf{Решение:}\\
Используя формулу для двойного угла, выразим: $\cos x = 2 \cos^2 (x/2) - 1$


$$\int\frac{1}{1+\cos x} dx = \int\frac{1}{2\cos^2 (x/2)} dx=\frac{1}{2} \int\sec^2 (x/2) dx = 2*\frac{1}{2} \tg(x/2) + C =  \tg(x/2) + C$$

\textbf{Ответ: $\tg(x/2) + C$}
\end{enumerate}

\item Посчитайте определенные интегралы
\begin{enumerate}
\item 
$
\int^2_1 x \ln x dx
$

\textbf{Решение:}\\
Интегрируем по частям:
$\int f dg = fg - \int gdf$ где, $f = \ln x, dg = xdx, df = 1/x dx, g = x^2 / 2$

$$
\int^2_1 x \ln x dx = 1/2*x^2 \ln x - 1/2 \int^2_1  x dx = \frac{1}{2} x^2 \ln x - \frac{x^2}{4}\Biggr|_{1}^{2} = 2\ln2 - 1 + 1/4 = 2\ln2 -3/4
$$

\textbf{Ответ: $2\ln2 -3/4$}

\item 
$
\int^{2\pi}_0 x \sin x dx
$

\textbf{Решение:}\\
Интегрируем по частям:
$\int f dg = fg - \int gdf$ где, $f =  x, dg = \sin x dx, df = dx, g = -\cos x$

$$
\int^{2\pi}_0 x \sin x dx
 = -x \cos x + \int^{2\pi}_0  \cos x dx = \sin{x} -x \cos{x}\Biggr|_{0}^{2\pi} =
$$
$$= sin(2\pi) - 2\pi cos(2\pi) - (sin(0) - 0 cos(0) ) = -2\pi$$

\textbf{Ответ: $-2\pi$}


\end{enumerate}

\item Найти предел $\lim_{x \to 0 }\frac{\int_{x}^{1} \frac{e^t}{t} dt}{\ln x}$

\textbf{Решение:}\\
Моё понимание, что интеграл в числителе будет равен разности двух интегральных показательных функций

$$
\lim_{x \to 0 }\frac{Ei(1) - Ei(x)}{\ln x}=
$$

Возьмем производную от верхней и нижней части 
$$= \lim_{x \to 0 }\frac{-e^x / x}{1/x} =\lim_{x \to 0 }-e^x = -1 $$

\textbf{Ответ: $-1$}

\item Найти площадь фигуры внутри кривой $\frac{x^2}{a^2}+\frac{y^2}{b^2}=1$

\textbf{Решение:}\\
Фигура симметрична относительно оси OX и OY. Поэтому оценим только верхнюю правую четверть площади. Выразим из формулы y и возьмем интеграл от 0 до a.

$$\int^{a}_0 b \sqrt{1-\frac{x^2}{a^2}} dx
$$
делаем замену $\sin t = x/a, dx = a \cos t dt$

$$\int^{\pi/2}_0 a b \sqrt{1-\sin^2t} \cos t dt =\int^{\pi/2}_0 a b \cos^2 t dt 
$$

Заменим через формулу двойного угла $\cos^2t = \frac{\cos2t +1 }{2}$

$$=\int^{\pi/2}_0 a b\frac{\cos2t +1 }{2} dt = 1/2 b a (1/2 \sin 2t + t)\Biggr|_{0}^{\pi/2} = 1/4 a b \pi$$

Общая площадь $a b \pi$

\textbf{Ответ: $a b \pi$}
\item Найти длину кривой $y = x^{3/2}$ от 0 до 4

\textbf{Решение:}\\
Запишем, что нужно считать

$$\int^{4}_0  \sqrt{1+(y')^2} dx=
$$


$y'(x) = 3/2 x^{0.5}, (y')^2 = \frac{9x}{4}$

$$=\int^{4}_0  \sqrt{1+\frac{9x}{4}} dx=
$$

заменим $u = \frac{9x}{4}+1, du = 9/4dx$


$$=-\frac{4}{9}\int^{10}_1 \sqrt{u}du= \frac{8u^{3/2}}{27}\Biggr|_{1}^{10} = \frac{8}{27} (10\sqrt{10}-1)$$


\textbf{Ответ: $ \frac{8}{27} (10\sqrt{10}-1)$}

\end{enumerate}
\end{document}